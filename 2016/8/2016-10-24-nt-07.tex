\input{../../main}

\begin{document}

\setphysstyle{ГЦФО 8}{Серия НТ-07}{24.10.2016}

\large

\taskpic[90pt]{Рычаг подвешен к системе блоков так, что точки подвеса
  делят его в отношении $a : b : c$ (см. рис.). Блоки, рычаг и нити
  невесомы, трения нет. Каково отношение масс грузов $m_1$ и $m_2$ ,
  если система находится в равновесии?}
{
  \begin{tikzpicture}[scale = 0.5]
    \draw[interface] (0, 6) -- (5, 6);
    \draw (2, 3) circle (0.5);
    \draw (3, 5) circle (0.5);
    \draw (3, 5) -- (3, 6);
    \draw (2, 3) -- (2, 2);
    \draw (2.5, 3) -- (2.5, 5);
    \draw (1.5, 3) -- (1.5, 6);
    \draw (3.5, 5) -- (3.5, 2);
    \draw (0.5, 2) -- (4.5, 2);
    \draw (0.3, 1.3) rectangle (0.7, 1.7);
    \draw (4.3, 1.3) rectangle (4.7, 1.7);
    \draw (0.5, 2) -- (0.5, 1.7) node[below left] {\scriptsize $m_1$};
    \draw (4.5, 2) -- (4.5, 1.7) node[below right] {\scriptsize $m_2$};
    \draw (1.2, 1.8) node[above] {\scriptsize $a$};
    \draw (2.7, 1.8) node[above] {\scriptsize $b$};
    \draw (4, 1.8) node[above] {\scriptsize $c$};
  \end{tikzpicture} 
}

\taskpic[90pt]{С помощью метода виртуальной работы определить силу
  натяжения каждой из пружинок на рисунке.\\ \textit{Указание:}
  попробуйте <<пошевелить>> каждую пружинку по отдельности.}
{
  \begin{tikzpicture}[scale = 0.5]
    \draw[interface] (1, 6) -- (4, 6);
    \draw (2, 4) circle (0.5);
    \draw (3, 3) circle (0.5);
    \draw (2, 4.5) -- (2, 4.7);
    \draw [spring] (2, 4.7) -- (2, 5.7);
    \draw (2, 5.7) -- (2, 6);
    \draw (1.5, 4) -- (1.5, 3);
    \draw [spring] (1.5, 3) -- (1.5, 1.3);
    \draw (1.5, 1.3) -- (1.5, 1);
    \draw[interface] (2, 1) -- (1, 1);
    \draw (2.5, 4) -- (2.5, 3);
    \draw (3.5, 3) -- (3.5, 6);
    \draw (3, 3) -- (3, 2);
    \draw (2.6, 1.3) rectangle (3.4, 2);
    \draw (3, 1.65) node {\scriptsize $m$};
  \end{tikzpicture}
}

\taskpic[90pt]{Тело поднимают с помощью наклонной плоскости и системы
  блоков (см. рис.). Какую минимальную силу $F$ нужно приложить, чтобы
  поднять тело массы $m$? Высота наклонной плоскости равна $H$, длина
  $L$. Блоки невесомые. Трением пренебречь.}
{
  \begin{tikzpicture}[scale = 0.5]
    \draw[interface] (0, 6) -- (6, 6);
    \draw (4, 5) circle (0.5);
    \draw (5, 3) circle (0.5);
    \draw (4.5, 5) -- (4.5, 3);
    \draw (5.5, 3) -- (5.5, 6);
    \draw[->] (5, 3) -- (5, 2) node[right] {\scriptsize $F$};
    \draw (4, 5) -- (4, 6);
    \draw (0, 2) -- (3, 2) -- (3, 4.5) -- (0, 2);
    \draw (1, 2.833333) -- ++(1, 0.833333) -- ++(-0.416665, 0.5) -- ++ (-1, -0.833333) -- ++(0.416665, -0.5);
    \draw (1.7916667, 3.9166663) -- (3.77, 5.45);
    \draw (1.32, 3.563333) node {\scriptsize $m$};
    \draw (1.8, 3.1) node {\scriptsize $L$};
    \draw (3.3, 3) node {\scriptsize $H$};
  \end{tikzpicture}
}

\taskpic[90pt]{На катушку намотаны две нити, за одну из них катушка
  подвешена к потолку, а ко второй нити подвешен груз. Вес катушки
  равен $m$, ее большой и малый радиусы равны $R$ и $r$
  соответственно. При какой массе груза $M$ катушка будет в
  равновесии?}
{
  \begin{tikzpicture}[scale = 0.5]
    \draw[interface] (0, 6) -- (4, 6);
    \draw (2, 3) circle (0.5);
    \draw (2, 3) circle (1);
    \draw (1.5, 3) -- (1.5, 6);
    \draw (1, 3) -- (1, 1);
    \draw (0.5, 0) rectangle (1.5, 1);
    \draw (1, 0.5) node {\scriptsize $M$}; 
    \draw [->] (2, 3) -- (2.5, 3);
    \draw (2.2, 3.2) node {\tiny $r$};
    \draw [->] (2, 3) -- (2.8, 2.35);
    \draw (2.15, 2.25) node {\tiny $R$};
    \draw (2.85, 1.8) node {\tiny $m$};
  \end{tikzpicture}
}

\task{Саша решил прокатить Катю на мотоцикле из Липовки в
  Демушкино. На пути из Липовки в Демушкино находится деревня
  Малиновка. Спустя $t_1 = 8\mbox{ мин}$ после выезда из Липовки Катя
  спросила: <<Какой путь мы проехали?>>. Саша ответил: <<Вдвое меньше,
  чем отсюда до Малиновки>>. Когда они проехали еще
  $L = 14\mbox{ км}$, Катя спросила: <<Сколько нам еще ехать до
  Демушкино?>>. Саша ответил: <<Вдвое больше, чем отсюда до
  Малиновки>>. Спустя $t_2 = 12\mbox{ мин}$ после этого они прибыли в
  Демушкино. Найдите скорость мотоцикла, считая ее постоянной и
  меньшей $60\mbox{ км/ч}$.}

\end{document}


%%% Local Variables: 
%%% mode: latex
%%% TeX-engine:xetex
%%% TeX-PDF-mode: t
%%% End:
