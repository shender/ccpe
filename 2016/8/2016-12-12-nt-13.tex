\documentclass[a4paper, 12pt]{article}
% math symbols
\usepackage{amssymb}
\usepackage{amsmath}
\usepackage{mathrsfs}
\usepackage{physsummer}


\usepackage{enumitem}
\usepackage[margin = 2cm]{geometry}

\tolerance = 1000
\emergencystretch = 0.74cm



\pagestyle{empty}
\parindent = 0mm

\begin{document}

\setphysstyle{ГЦФО 8}{Серия НТ-13}{12.12.2016}

\large

\task{Всем известный дуб у Лукоморья имеет форму идеального конуса
  высотой $h$ и радиусом основания $r$. При помощи метода виртуальных
  перемещений определите силу натяжения златой цепи массой $m$,
  висящей на сказочном дубе.\\
  {\itshape Примечание}: длина окружности радиуса $r$ равна
  $2 \pi r$.}

\taskpic[3cm]{В системе блоков, изображённой на рисунке, балку
  удерживают в горизонтальном положении так, что пружина не
  растянута. Пружина имеет жёсткость $k = 20\mbox{ Н/м}$, масса балки
  $m = 7{,}5\mbox{ кг}$, нити и блоки идеальные. Балку отпускают, и
  система вновь приходит в равновесие. Определите, на какую длину
  растянется пружина, а также на какую длину сместится балка
  относительно начального положения. Считать, что в положении
  равновесия балка снова горизонтальна.}
{\begin{tikzpicture}[scale = 0.5]
    \draw[interface] (1, 6) -- (4, 6);
    \draw (2, 4) circle (0.5);
    \draw (3, 3) circle (0.5);
    \draw (2, 4.5) -- (2, 4.7);
    \draw (2, 4) -- (2, 5.7);
    \draw (2, 5.7) -- (2, 6);
    \draw (1.5, 4) -- (1.5, 1.3);
    \draw (1.5, 1.3) -- (1.5, 1);
    \draw (2.5, 4) -- (2.5, 3);
    \draw (3.5, 3) -- (3.5, 6);
    \draw (3, 3) -- (3, 2.4);
    \draw [spring] (3, 2.4) -- (3, 1);
    \draw (1, 0.5) rectangle (3.5, 1);
    \draw (2.3, 0.7) node[below] {\small $m$};
   \draw (3.5, 1.8) node {\scriptsize $k$};
\end{tikzpicture} }

\taskpic[3cm]{В изображённой на рисунке системе к двум невесомым
  поршням гидравлического пресса на невесомой нерастяжимой нити
  подвешен подвижный блок массой $m_1 = 25\mbox{ кг}$ и объёмом $V_1 =
  10\mbox{ л}$. К блоку прикреплён груз массой $m_2 = 170\mbox{ кг}$ и
  объёмом $V_2 = 15\mbox{ л}$. Поршни вначале закреплены и находятся
  на высоте $H = 3\mbox{ м}$. Их отпустили, и система пришла к
  равновесию. На какой высоте теперь находятся поршни? Площади колен
  равняются $S_1 = 0{,}7\mbox{ м}^2$ и $S_2 = 0{,}2\mbox{ м}^2$
  соответственно. Плотность воды $\rho = 1000\mbox{
    кг/м}^3$. Считайте, что груз не достаёт до дна.} 
{\begin{tikzpicture}[scale = 0.5]
	\draw (0, 5) -- (0, 4);
	\draw (4, 5) -- (4, 4);
	\draw[fill=blue!20] (0, 3.8) rectangle (4, -1);
    \draw[pattern=north east lines] (0, 4) rectangle (2.5, 3.8);
    \draw[pattern=north east lines] (4, 4) rectangle (3.2, 3.8);
    \draw[fill = white] (2.5, 5) -- (2.5, 2.5) -- (3.2, 2.5) -- (3.2, 5); 
    \draw[fill = white] (2.9, 1) circle (0.7);
    \draw (2.2, 1) -- (2.2, 3.8);
    \draw (3.6, 1) -- (3.6, 3.8);
    \draw[fill = white] (2.6, -0.6) rectangle (3.2, 0);
    \draw (2.9, 1) -- (2.9, -0.3);
\end{tikzpicture} }


\end{document}

%%% Local Variables: 
%%% mode: latex
%%% TeX-engine:xetex
%%% TeX-PDF-mode: t
%%% End:
