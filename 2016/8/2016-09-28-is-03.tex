\documentclass[a4paper, 12pt]{article}
% math symbols
\usepackage{amssymb}
\usepackage{amsmath}
\usepackage{mathrsfs}
\usepackage{physsummer}


\usepackage{enumitem}
\usepackage[margin = 2cm]{geometry}

\tolerance = 1000
\emergencystretch = 0.74cm



\pagestyle{empty}
\parindent = 0mm

\begin{document}

\setphysstyle{ГЦФО 8}{Серия Ш-03}{28.09.2016}

\setcounter{notask}{9}

\taskpic[3cm]{ Два трамвая, двигавшихся навстречу, проходят прямолинейный
  участок путей длиной 480 м со скоростями 10 м/с. Кондукторы ходят от
  задних площадок к передним и обратно со скоростями 2 м/с. Расстояния
  между площадками в вагонах 12 м. В середине участка кондукторы
  поравнялись, находясь на задних площадках. Постройте графики
  зависимостей пути и скорости кондукторов относительно земли от
  времени от момента встречи кондукторов до момента достижения ими
  концов участка. Через сколько времени кондукторы окажутся на концах
  участка, если за начальный момент принять момент встречи
  кондукторов?  }
{
  \begin{tikzpicture}
    \draw[thick] (0,1.5) -- (0,-1.5);
    \draw[thick] (0.2,1.5) -- (0.2,-1.5); 
    \draw[thick] (2,1.5) -- (2,-1.5);
    \draw[thick] (2.2,1.5) -- (2.2,-1.5);
    \draw[thick,fill=white] (-0.1,0.3) rectangle (0.3,-0.8);
    \draw[*->,thick] (0.1,0.2) -- ++(0,-0.8);
    \draw[thick,fill=white] (1.9,-0.1) rectangle ++(0.4,1.1);
    \draw[*->,thick] (2.1,0) -- ++(0,0.8);
  \end{tikzpicture}
}
% Район-2003, 8 класс

\task{ Колонна машин движется по дороге, строго соблюдая определённый
  скоростной режим --- зависимость скорости машины от её положения $x$
  на дороге представлена на графике сплошной линией. Известно, что
  машины начинали своё движение с интервалом в $\tau=10$ с. В
  некоторый момент всем машинам одновременно поступило сообщение об
  ухудшении погодных условий, в соответствии с которым они должны
  изменить свой скоростной режим на другой --- изображённый
  прерывистой линией. Какой максимальный временной интервал будет
  наблюдаться между машинами, приходящими в конечный пункт? }

\begin{figure}[h]
  \centering
  \begin{tikzpicture}[/pgfplots/axis labels at tip/.style={ xlabel
      style={at={(current axis.right of origin)}, yshift=2 ex,
        anchor=east,fill=white}, ylabel style={at={(current axis.above
          origin)}, yshift=1.5ex, anchor=center}}]
    \begin{axis}[
      width=8cm,
      xmin=0,xmax=6,ymin=0,ymax=80,
      axis x line=bottom,
      axis y line=middle,
      axis labels at tip,
      xlabel={$x$, км},
      ylabel={$v$, км/ч},
      minor xtick={0.25,0.5,0.75,...,6},
      minor ytick={5,10,15,...,80},
      xtick={0,1,2,...,6},
      ytick={0,20,40,60},
      xticklabels={0,1,2,...,6},
      yticklabels={0,20,40,60},
      tick label style={font=\small},
      label style={font=\small},
      grid=both
      ]
      \addplot[very thick,red,mark=none] coordinates { (0,55)
        (0.75,55) (0.75,65) (1.75,65) (1.75,45) (3,45) (3,60) (4,60)
        (4,55) (5,55) (5,70) (6,70) };
      \addplot[very thick,dashed,blue,mark=none] coordinates { (0,40) (0.75,40)
        (0.75,45) (1.75,45) (1.75,35) (3,35) (3,40) (4,40) (4,50) (6,50)  };
    \end{axis}
  \end{tikzpicture}%
\end{figure}
% Город-2007, 8 класс

\task{ Два велосипедиста одновременно выезжают навстречу друг другу из
  деревень Егорово и Серово, находящихся на расстоянии $L=10$ км друг
  от друга. Каждый планирует ехать со скоростью $V=20$ км/ч и,
  достигнув противоположной деревни, сразу повернуть обратно. Но вдоль
  дороги всё время дует ветер, скорость и направление которого
  постоянны. При движении по ветру скорость увеличивается на столько
  же, на сколько уменьшается при движении против ветра. Велосипедист,
  который сначала ехал по ветру, достигнув противоположной деревни,
  сразу повернул назад, а велосипедист, который сначала ехал против
  ветра, задержался в противоположной деревне, чтобы отдохнуть, и
  только потом поехал обратно. Известно, что велосипедисты встречались
  в точках \textbf{A} и \textbf{B}, находящихся на расстоянии $L_A =
  2$ км и $L_B=6$ км от Егорово. Найдите времена движения обоих
  велосипедистов. Найдите также время, которое потратил на отдых
  уставший велосипедист. }
% Москва-2007, 8 класс

\end{document}

%%% Local Variables: 
%%% mode: latex
%%% TeX-engine:xetex
%%% TeX-PDF-mode: t
%%% End:
