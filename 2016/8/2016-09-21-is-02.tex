\documentclass[a4paper, 12pt]{article}
% math symbols
\usepackage{amssymb}
\usepackage{amsmath}
\usepackage{mathrsfs}
\usepackage{physsummer}


\usepackage{enumitem}
\usepackage[margin = 2cm]{geometry}

\tolerance = 1000
\emergencystretch = 0.74cm



\pagestyle{empty}
\parindent = 0mm

\begin{document}

\setphysstyle{ГЦФО 8}{Серия Ш-02}{21.09.2016}

\large

\task{ Экспериментатор Глюк на большом лабораторном столе проводил
  испытания модели вездехода. Координатную ось $X$ он направил вдоль
  длинного края стола. Зависимости координаты $x(t)$ и пройденного
  пути $s(t)$ от времени приведены на графиках. Опишите характер
  движения модели вездехода. С какой максимальной скоростью двигался
  вездеход? На каком расстоянии друг от друга находятся начальные и
  конечные точки его движения? }
\begin{figure}[h]
  \centering
  \subfloat[][]{
    \begin{tikzpicture}[/pgfplots/axis labels at tip/.style={ xlabel
        style={at={(current axis.right of origin)}, yshift=2 ex,
          anchor=east,fill=white}, ylabel style={at={(current
            axis.above origin)}, yshift=1.5ex, anchor=center}}]
    \begin{axis}[
      width=8cm,
      xmin=0,xmax=130,ymin=20,ymax=75,
      axis x line=bottom,
      axis y line=middle,
      axis labels at tip,
      xlabel={$t$, с},
      ylabel={$x$, см},
      xtick={0,10,...,120},
      ytick={20,30,...,70},
      xticklabels={0,10,20,...,120},
      yticklabels={20,30,...,70},
      tick label style={font=\scriptsize},
      label style={font=\small},
      grid=both
      ]
      \addplot[very thick,red,mark=none] coordinates {(0,30) (40,70)
        (60,70) (120,50)};
    \end{axis}
  \end{tikzpicture}}%
\hspace{1cm}
  \subfloat[][]{
    \begin{tikzpicture}[/pgfplots/axis labels at tip/.style={ xlabel
        style={at={(current axis.right of origin)}, yshift=2 ex,
          anchor=east,fill=white}, ylabel style={at={(current
            axis.above origin)}, yshift=1.5ex, anchor=center}}]
    \begin{axis}[
      width=8cm,
      xmin=0,xmax=130,ymin=0,ymax=115,
      axis x line=bottom,
      axis y line=middle,
      axis labels at tip,
      xlabel={$t$, с},
      ylabel={$s$, см},
      xtick={0,10,...,120},
      ytick={0,20,...,100},
      xticklabels={0,10,20,...,120},
      yticklabels={0,20,...,100},
      tick label style={font=\scriptsize},
      label style={font=\small},
      grid=both
      ]
      \addplot[very thick,red,mark=none] coordinates {(0,0) (40,40)
        (60,80) (120,100)};
    \end{axis}
  \end{tikzpicture} 
  }

\end{figure}
% Максвелл-2016, 8 класс

\task{ Колонна бегунов имеет скорость $v$ и длину $l$. Навстречу
  бегунам бежит тренер со скоростью $u$ ($u<v$). Поравнявшись с
  тренером, каждый бегун поворачивает в противоположном направлении и
  бежит со скоростью $v$. Нарисуйте графики зависимостей координат от
  времени для колонны спортсменов и тренера в системе отсчёта а)
  земли; б) колонны.}
% Шапиро-Бодик-1, стр. 11

\task{ По прямой на наблюдателя со скоростью $v$ движется источник
  сигналов. Скорость сигналов $w > v$. Источник испускает сигналы
  через промежутки времени $t_0$. Нарисуйте графики $s(t)$ для
  источника и сигналов. Что можно сказать о промежутках времени,
  разделяющих приходы сигналов к наблюдателю? }

\task{ Велосипедист едет по дороге и через каждые 6 секунд проезжает
  мимо столба линии электропередачи. Увеличив скорость на некоторую
  величину $\Delta v$, велосипедист стал проезжать мимо столбов через
  каждые 4 секунды. Через какой промежуток времени он будет проезжать
  мимо столбов, если увеличит скорость ещё
  на такую же величину?
}
% Район 2005, 8 класс


\end{document}

%%% Local Variables: 
%%% mode: latex
%%% TeX-engine:xetex
%%% TeX-PDF-mode: t
%%% End:
