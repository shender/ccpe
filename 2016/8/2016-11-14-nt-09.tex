\input{../../main}

\begin{document}

\setphysstyle{ГЦФО 8}{Серия НТ-09}{14.11.2016}

\large

\task{В теплоизолированном сосуде (калориметре) было некоторое
  количество льда с температурой $T_1 = -10^{\circ}$C. После того, как
  в калориметр добавили $m = 0{,}5\mbox{ кг}$ воды с температурой
  $T_2 = 60^{\circ}$C, в сосуде установилась температура
  $T = +10^\circ$C. Определите, сколько льда изначально было в
  калориметре. Теплоёмкость воды
  $c_{\mbox{в}} = 4200\mbox{ Дж/кг}\cdot{}^{\circ}$C, теплоёмкость
  льда $c_{\mbox{л}} = 2100\mbox{ Дж/кг}\cdot{}^{\circ}$C, теплота
  плавления льда $\lambda = 340\mbox{ кДж/кг}$.}

\task{В бассейн по трубе, в которой установлен нагреватель мощностью
  $P = 1\mbox{ МВт}$, подается вода из резервуара. Температура воды в
  резервуаре $T_p = 5^\circ$C. В первый раз пустой бассейн
  заполняется за время $\tau = 21\mbox{ мин}$, при этом температура
  воды после заполнения $T_1 = 20^\circ$C. Во второй раз в бассейне
  было изначально некоторое количество воды при температуре
  $T_0 = 15^\circ$C. Оставшуюся часть заполняли также время
  $\tau = 21\mbox{ мин}$. Температура воды после заполнения оказалась
  $T_2 = 25^\circ$C. Сколько воды первоначально было в бассейне во
  втором случае? Остыванием воды в бассейне пренебречь. Удельная
  теплоёмкость воды $c_{\mbox{в}} = 4200\mbox{ Дж/кг}\cdot{}^\circ$C.}

\task{В трёх калориметрах находится по $M = 20\mbox{ г}$ воды
  одинаковой температуры. В калориметры погружают льдинки, также
  имевшие одинаковые температуры (но другие): в первый --- льдинку
  массой $m_1 = 10\mbox{ г}$, во второй --- массой
  $m_2 = 20\mbox{ г}$, в третий --- массой $m_3 = 40\mbox{ г}$. Когда
  в калориметрах установилось равновесие, оказалось, что масса первой
  льдинки стала равной $m'_1 = 9\mbox{ г}$, а масса второй льдинки
  осталась прежней. Какой стала масса третьей льдинки $m'_3$ ?}

\end{document}


%%% Local Variables: 
%%% mode: latex
%%% TeX-engine:xetex
%%% TeX-PDF-mode: t
%%% End:
