\input{../../main}

\begin{document}

\setphysstyle{ГЦФО 8}{Серия НТ-03}{26.09.2016}

\Large

\task{ Сережа ездит в школу на маршрутке, которая идет по прямой
  дороге со скоростью $v$. Когда Сережа был на расстоянии $S$ от
  дороги, он заметил маршрутку на расстоянии $L$ от себя. С какой
  наименьшей скоростью ему нужно бежать, чтобы успеть перехватить
  маршрутку? Маршрутка может остановиться в любом месте, но не будет
  ждать пока Сережа добежит.}

\task{ Мальчик поднимается в гору со скоростью $1\mbox{ м/c}$. Когда
  до вершины остается идти $100\mbox{ м}$, мальчик отпускает собаку, и
  она начинает бегать между мальчиком и вершиной горы. Собака бежит к
  вершине со скоростью $3\mbox{ м/c}$, а возвращается к мальчику со
  скоростью $5\mbox{ м/c}$. Какой путь успеет пробежать собака до
  того, как мальчик достигнет вершины?}

\task{ Упругий мячик при ударе об стенку отскакивает назад с такой же
  по величине скоростью. С какой скоростью полетит такой мячик, если
  по неподвижному мячику ударить ногой со скоростью $v$?}

\task{ Две космические ракеты сближаются со скоростью
  $8000\mbox{ км/ч}$. С одной ракеты через каждые $20\mbox{ минут}$
  посылают на другую почтовые контейнеры со скоростью
  $8000\mbox{ км/ч}$. Сколько сообщений получит экипаж второй ракеты
  за час?}

\end{document}


%%% Local Variables: 
%%% mode: latex
%%% TeX-engine:xetex
%%% TeX-PDF-mode: t
%%% End:
