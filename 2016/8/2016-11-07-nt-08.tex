\documentclass[a4paper, 12pt]{article}
% math symbols
\usepackage{amssymb}
\usepackage{amsmath}
\usepackage{mathrsfs}
\usepackage{physsummer}


\usepackage{enumitem}
\usepackage[margin = 2cm]{geometry}

\tolerance = 1000
\emergencystretch = 0.74cm



\pagestyle{empty}
\parindent = 0mm

\begin{document}

\setphysstyle{ГЦФО 8}{Серия НТ-08}{07.11.2016}

\large

\task{В двух одинаковых бочках находится одинаковое количество
  воды. Температура воды в первой бочке $T_1 = 20^\circ$C, а во
  второй бочке $T_1 = 60^\circ$C. Из первой бочки перелили некоторое
  количество воды во вторую, и в ней установилась температура
  $T = 50^\circ$C. Затем из второй бочки перелили такое же количество
  воды в первую так, что воды в бочках снова стало поровну. Какая
  температура установится в первой бочке? Всеми потерями тепла во
  внешнюю среду и механической работой, совершённой при переливании
  воды, пренебречь.}

\task{Три одинаковых стакана наполовину заполнены водой разной
  температуры. Известно, что температура воды в третьем стакане на
  $12^\circ$C больше, чем во втором. В первом опыте сначала всю воду
  из первого стакана выливают во второй, затем из второго стакана
  доливают воду в третий стакан, так что он становится полным. После
  этого воду из стаканов выливают и наполняют их так же, как перед
  первым опытом. Во втором опыте сначала всю воду из первого стакана
  выливают в третий, после чего половину воды из третьего стакана
  отливают в первый стакан и освободившееся место заполняют водой из
  второго стакана. Насколько конечная температура воды в третьем
  стакане в первом опыте больше, чем во втором? Вода в стаканах
  смешивается быстро, теплообменом воды со стаканами и окружающей
  средой в течение опытов можно пренебречь.}

\task{В цилиндрический стакан налита вода до уровня
  $h_0 = 10\mbox{ см}$ при температуре $T_0 = 0^\circ$C. В стакан
  бросают алюминиевый шарик, вынутый из другого сосуда с водой,
  кипящей при температуре $T_k = 100^\circ$C. При этом уровень воды
  повышается на $x = 1\mbox{ см}$. Какой будет установившаяся
  температура в стакане? Удельные теплоёмкости воды и алюминия
  $C_{\mbox{в}} = 4200\mbox{ Дж/кг}\cdot {}^\circ$C и
  $C_{\mbox{a}} = 920\mbox{ Дж/кг}\cdot {}^\circ$C, плотности воды и алюминия
  $\rho_{\mbox{в}} = 1000\mbox{ кг/м}^3$ и
  $\rho_a = 2700\mbox{ кг/м}^3$.}


\end{document}


%%% Local Variables: 
%%% mode: latex
%%% TeX-engine:xetex
%%% TeX-PDF-mode: t
%%% End:
