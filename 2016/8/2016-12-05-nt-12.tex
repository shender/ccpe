\input{../../main}

\begin{document}

\setphysstyle{ГЦФО 8}{Серия НТ-12}{05.12.2016}

\Large

\task{Андрей очень любит пить чай, но строго определенной температуры,
  ровно $80^\circ$C. Однажды он пришел в гости к Кате. К неё есть
  чайник с заваркой комнатной температуры $30^\circ$C и только что
  вскипевший чайник с кипятком. Сколько заварки и сколько кипятка Катя
  должна налить в кружку, чтобы получить $350\mbox{ мл}$ чая?}

\task{Два сплошных шара из одинакового металла, радиусами $R$ и $2R$,
  были погружены в кипящую воду. После того, как маленький шар вынули
  из кипятка и опустили в ванну с холодной водой со льдом, он за $20$
  секунд остыл до $50^\circ$C. За какое время вынутый из кипятка
  большой шар остынет в холодной ванне до $50^\circ$C? За какое
  примерно время маленький шар в ванне остынет от $50^\circ$C до
  $25^\circ$C?}
\end{document}

%%% Local Variables: 
%%% mode: latex
%%% TeX-engine:xetex
%%% TeX-PDF-mode: t
%%% End: