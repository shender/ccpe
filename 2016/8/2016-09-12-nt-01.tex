\input{../../main}

\begin{document}

\setphysstyle{ГЦФО 8}{Серия НT-01}{12.09.2016}

\task{Катер, двигаясь вниз по течению, затратил времени в $n = 3$ раза
  меньше, чем на обратный путь. Определите, с какими скоростями
  двигался катер по течению и против течения, если средняя скорость
  катера на всем пути составила $u = 3 \mbox{км/ч}$. Известно, что
  скорость катера относительно воды постоянная.. }

\task{По реке со скоростью $v$ плывут мелкие льдины, которые
  равномерно распределяются по поверхности воды, покрывая $1/n$ часть
  поверхности реки. В некотором месте реки образовался затор. В заторе
  льдины полностью покрывают поверхность воды, не нагромождаясь друг
  на друга. С какой скоростью движется граница затора?}

\task{ Пешеход, велосипедист и мотоциклист двигались по шоссе в одну
  сторону. В момент, когда велосипедист и пешеход были в одном месте,
  мотоциклист отставал от них на 6 км. А когда мотоциклист догнал
  велосипедиста, пешеход отставал от них на 3 км. На сколько
  километров велосипедист обогнал пешехода в тот момент, когда
  пешехода нагнал мотоциклист?}

\task{Три мушкетера и д’Артаньян скачут из Тура в Булонь. Герои
  стартовали в разное время, но скачут с постоянными скоростями, по
  одной дороге и в одном направлении. Портос встретился с Атосом в
  11.10, с д’Артаньяном ровно в полдень, а с Арамисом – в половине
  первого. Известно, что Атос и Арамис прибыли в Булонь одновременно,
  в 14.10. Атос и д’Артаньян потратили одинаковое время на
  дорогу. Когда встретились Арамис и д’Артаньян?}

\task{Туннель, соединяющий Институт Галактических Исследований и
  Университет Межпланетного Культурного Обмена, состоит из трех
  участков. Время движения по участку $AB$ занимает
  $t_1 = 2 \mbox{ минуты}$, а по участкам $BC$ и $CD$
  $t_2 = t_3 = 4 \mbox{ минуты.}$ К 2510-му году решено
  модернизировать туннель, установив в нем две пары кабин
  нуль-транспортировки, позволяющих мгновенно перемещаться из одной
  точки в другую большому количеству людей. Первая пара кабин
  позволяет мгновенно перемещаться между точками $A$ и $C$, а вторая
  пара --- между точками $B$ и $D$ (см. рис). На каждом участке
  туннеля $AB$, $BC$ и $CD$ одновременно может находиться не более
  $N = 1000$ представителей разумных расс. Какое максимальное
  количество разумных существ будет способно пройти из $A$ в $D$ за
  час после модернизации туннеля?}

\begin{figure}[h]
  \centering
  \begin{tikzpicture}[scale = 2]
	\draw(0, 0) -- ++(3, 0);
	\draw[rounded corners = 15, <->] (0, 0)-- ++(1, -1) -- ++(1, 1);
	\draw[rounded corners = 15, <->] (1, 0)-- ++(1, 1) -- ++(1, -1);
	\draw (0, 0.2) node  {$A$};
	\draw (1, 0.2) node  {$B$};
	\draw (2, 0.2) node  {$C$};
	\draw (3, 0.2) node  {$D$};
	\draw (0.5, -0.2) node  {$t_1$};
	\draw (1.5, -0.2) node  {$t_2$};
	\draw (2.5, -0.2) node  {$t_3$};
   \end{tikzpicture}
\end{figure}

\end{document}

%%% Local Variables: 
%%% mode: latex
%%% TeX-engine:xetex
%%% TeX-PDF-mode: t
%%% End:
