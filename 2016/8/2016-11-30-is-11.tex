\documentclass[a4paper, 12pt]{article}
% math symbols
\usepackage{amssymb}
\usepackage{amsmath}
\usepackage{mathrsfs}
\usepackage{physsummer}


\usepackage{enumitem}
\usepackage[margin = 2cm]{geometry}

\tolerance = 1000
\emergencystretch = 0.74cm



\pagestyle{empty}
\parindent = 0mm

\begin{document}

\setphysstyle{ГЦФО 8}{Серия Ш-11}{30.11.2016}

\setcounter{notask}{32}

\Large

\task{ На горизонтальную поверхность льда при температуре
  $T_1=0^{\circ}$C кладут однорублёвую монету, нагретую до температуры
  $T_2=50^{\circ}$C. Монета проплавляет лёд и опускается в
  образовавшуюся лунку. На какую часть своей толщины она погрузится в
  лёд? Удельная теплоёмкость материала монеты
  $C=380\mathrm{~Дж/кг}\cdot^{\circ}$C, плотность его
  $\rho=8{,}9 \mathrm{~г/см}^3$, удельная теплота плавления льда
  $\lambda = 3{,}4\cdot10^5$ Дж/кг, плотность льда
  $\rho_0=0{,}9 \mathrm{~г/см}^3$. }
% МФО 1986-2007, 2.9

\task{ В двух калориметрах налито по 200 г воды --- при температурах
  $+30^{\circ}$C и $+40^{\circ}$C. Из <<горячего>> калориметра
  зачерпывают 50 г воды, переливают в <<холодный>> и
  перемешивают. Затем из <<холодного>> калориметра переливают 50 г
  воды в <<горячий>> и снова перемешивают. Сколько раз нужно перелить
  такую же порцию воды туда-обратно, чтобы разность температур воды в
  калориметрах стала меньше $1^{\circ}$C? Потерями тепла в процессе
  переливаний и теплоёмкостью калориметра пренебречь. }
% МФО 1986-2007, 2.6

\task{ Имеется сосуд с небольшим отверстием у дна. В сосуд помещён
  большой кусок кристаллического льда при температуре
  $T_0=0^{\circ}$C. Сверху на лёд падает струя воды, её температура
  $T_1=20^{\circ}$C, а расход $q=1$ г/с. Найдите расход воды,
  вытекающей из сосуда, если её температура
  $T=3^{\circ}$C. Теплообменом с окружающим воздухом и с сосудом можно
  пренебречь. Удельная теплоёмкость воды ${C=4{,}2\mathrm{~
    кДж/кг}\cdot^{\circ}}$C, удельная теплота плавления льда
  ${\lambda=340}$~Дж/г. Вода в сосуде не накапливается. }
% МФО 1986-2007, 2.11


\end{document}

%%% Local Variables: 
%%% mode: latex
%%% TeX-engine:xetex
%%% TeX-PDF-mode: t
%%% End:
