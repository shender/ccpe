\documentclass[a4paper, 12pt]{article}
% math symbols
\usepackage{amssymb}
\usepackage{amsmath}
\usepackage{mathrsfs}
\usepackage{physsummer}


\usepackage{enumitem}
\usepackage[margin = 2cm]{geometry}

\tolerance = 1000
\emergencystretch = 0.74cm



\pagestyle{empty}
\parindent = 0mm

\begin{document}

\setphysstyle{ГЦФО 8}{Серия НТ-11}{28.11.2016}

\large

\task{Термос залили горячим чаем с температурой $80^\circ$C и плотно
  закрыли. На следующий день температура в термосе стала равна
  $50^\circ$C. Тогда чай из термоса выпили, а термос снова заполнили
  горячим чаем чаем с температурой $80^\circ$C, но только
  наполовину. Пренебрегая теплоёмкостью термоса вычислите, через какое
  время чай остынет до $50^\circ$C.}

\task{Катя в походе сварила большой котел супа. Поскольку суп еще
  очень горячий, она предлагает подождать $20$ минут, пока суп остынет
  до комфортной температуры. Но очень голодная Ксюша не хочет так
  долго ждать, поэтому она разлила весь суп по $8$ мискам. Миски имеют
  такую же форму, как и котел, но меньше по всем размерам в $2$
  раза. Через сколько времени суп остынет в мисках?}

\task{Маша прочитала на этикетке, что энергетическая ценность
  содержимого одной бутылки газировки равна $Q = 19642$
  калории. Сколько льда при температуре $0^\circ$C она должна добавить
  в газировку, перед тем как выпить ее, чтобы не потолстеть (не
  получить калорий)? Начальная температура газировки $20^\circ$C,
  температура Маши $36{,}6^\circ$C. Теплоёмкость газировки, которую
  собирается выпить Маша, $c_{\mbox{г}} = 1{,}8 \mbox{
    кДж/кг}\cdot^\circ$C. Удельная теплота плавления льда
  $\lambda = 333 \mbox{ кДж/кг}$, удельная теплоёмкость воды
  $c_{\mbox{в}} = 4200 \mbox{ Дж/кг}\cdot^\circ$C. Одна калория ---
  теплота, необходимая для того, чтобы нагреть один грамм воды на один
  градус.}


\end{document}


%%% Local Variables: 
%%% mode: latex
%%% TeX-engine:xetex
%%% TeX-PDF-mode: t
%%% End:
