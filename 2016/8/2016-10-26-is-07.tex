\input{../../main}

\begin{document}

\setphysstyle{ГЦФО 8}{Серия Ш-07}{26.10.2016}

\large
\setcounter{notask}{21}

\task{ К коромыслу равноплечих весов подвешены два груза одинаковой
  массы. Если каждый из грузов поместить в жидкости с плотностями
  $\rho_1$ и $\rho_2$, то равновесие сохранится. Найти отношение
  плотностей грузов.  }
% НГУ-1, 1.163

\task{ Два шара одинакового объёма, но разной плотности закреплены на
  концах стержня, шарнирно подвешенного в центре. Шарнир находится на
  поверхности воды, при этом шар погружается в воду на три четверти
  своего объёма, а другой --- на одну четверть. Найти плотность более
  тяжёлого шара $\rho_2$, если плотность лёгкого шара равна
  $\rho_1$. Плотность воды равна $\rho$. }
% НГУ-1, 1.164

\task{ Свинцовый и алюминиевый шарики одинакового радиуса $r$ связаны
  невесомой и нерастяжимой нитью, длина которой намного больше размера
  шариков. Шарики опустили в сосуд с глицерином. После этого они
  пришли в движение с нулевой начальной скоростью. Сила сопротивления
  движению шариков пропорциональна их скорости, причём коэффициент
  пропорциональности одинаков для обоих шариков. Найдите силу
  натяжения нити при установившейся скорости движения
  шариков. Плотности алюминия и свинца $\rho_1$ и $\rho_2$.  }
% Манида, стр. 29, задача 24

\vspace{1cm}

\textit{Указание.} Объём шара
  радиуса $r$ равен $V = 4 \pi r^3/3$.

\end{document}


%%% Local Variables: 
%%% mode: latex
%%% TeX-engine:xetex
%%% TeX-PDF-mode: t
%%% End:
