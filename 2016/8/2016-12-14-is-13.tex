\documentclass[a4paper, 12pt]{article}
% math symbols
\usepackage{amssymb}
\usepackage{amsmath}
\usepackage{mathrsfs}
\usepackage{physsummer}


\usepackage{enumitem}
\usepackage[margin = 2cm]{geometry}

\tolerance = 1000
\emergencystretch = 0.74cm



\pagestyle{empty}
\parindent = 0mm

\begin{document}

\setphysstyle{ГЦФО 8}{Серия Ш-13}{14.12.2016}

\setcounter{notask}{38}

\Large

\task{ В прямоугольный высокий сосуд налита жидкость плотности
  $\rho$. В одной из стенок у дна сосуда имеется прямоугольное
  отверстие высоты $h$, в которое вдвинута на расстояние $l$ невесомая
  пробка того же сечения. При какой высоте уровня жидкости над пробкой
  жидкость не сможет её вытолкнуть? Коэффициент трения пробки о дно
  сосуда равен $k$. Атмосферное давление равно $p_0$. Трением пробки о
  стенки сосуда пренебречь. }
% НГУ-1, 1.150

\taskpic[6cm]{ Планка массой $m$ и два одинаковых груза массой $2m$ с
  помощью лёгких нитей прикреплены к двум блокам. Система находится в
  равновесии. Определите силы натяжения нитей и силы, с которыми
  подставка действует на грузы. Трения в осях блоков нет. }
{
  \begin{tikzpicture}
    \draw[very thick] (0.5,0.2) -- (0.5,2.5);
    \draw[very thick] (1,2.5) circle (0.5cm);
    \draw[very thick] (1.5,2.5) -- (1.5,1);
    \draw[very thick] (1.3,1) rectangle (1.7,0.2) node[right=0.2cm,midway,blue]
    {\small $2m$};
    \draw[very thick] (3,0.2) -- (3,2.5);
    \draw[very thick] (3.5,2.5) circle (0.5cm);
    \draw[very thick] (4,2.5) -- (4,1);
    \draw[very thick] (3.8,1) rectangle (4.2,0.2)
    node[right=0.2cm,midway,blue] {\small $2m$};
    \draw[very thick] (1,2.5) -- (1,3.5);
    \draw[very thick] (3.5,2.5) -- (3.5,3.5);
    \draw[interface,thick] (0,3.5) -- (5,3.5);
    \draw[thick] (-0.5,0) rectangle ++(5.5,0.2) node[below=0.2cm,midway,blue]
    {$m$};
    \foreach \x in {0,0.5,...,5} { \draw (\x,0) -- (\x,0.2);}; 
  \end{tikzpicture}
}
% Максвелл-2015-8

\taskpic[5cm]{ Четыре одинаковых ледяных бруска длиной $L$ сложены так, как
  показано на рисунке. Каким может быть максимальное расстояние $d$,
  при условии, что все бруски расположены горизонтально? Считайте что
  бруски гладкие (между ними нет трения), и что сила тяжести приложена
  к центру соответствующего бруска. }
{
  \begin{tikzpicture}
    \draw[thick, interface] (4.5,0) -- (0,0);
    \draw[thick,fill=gray!70] (1.5,0) rectangle ++(1.5,0.5);
    \draw[thick,fill=gray!70] (0.5,0.5) rectangle ++(1.5,0.5);
    \draw[thick,fill=gray!70] (2.5,0.5) rectangle ++(1.5,0.5);
    \draw[thick,fill=gray!70] (1.5,1) rectangle ++(1.5,0.5);
    \draw[thick,blue,<->] (1.5,2) -- ++(1.5,0)
    node[midway,above,blue] {$L$};
    \draw[thick,blue,<->] (2,0.75) -- ++(0.5,0);
    \draw[thick,blue,->] (3,-0.5) node[right] {$d$} to[out=180,in=270] (2.25,0.7); 
  \end{tikzpicture}
}
% Максвелл-2013-8

\end{document}

%%% Local Variables: 
%%% mode: latex
%%% TeX-engine:xetex
%%% TeX-PDF-mode: t
%%% End:
