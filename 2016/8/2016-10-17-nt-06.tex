\input{../../main}

\setphysstyle{ГЦФО 8}{Серия НТ-06}{17.10.2016}

\begin{document}

\Large

\task{ На пружине жесткостью $k$ висит груз массой $m$. Груз оттянули
  вниз на расстояние $\Delta x$ от положения равновесия и
  отпустили. Какая при этом была произведена работа? До какой
  максимальной скорости разгонится груз? На какой высоте относительно
  положения равновесия груз остановится?}

\task{ Два тела массами $m_1$ и $m_2$ лежат на горизонтальной
  поверхности с коэффициентом трения $\mu$ и соединены изначально
  недеформированной пружиной жесткости $k$. На первое тело действуют с
  постоянной силой $F$. При каком минимальном значении этой силы
  второе тело сдвинется с места?}

\end{document}


%%% Local Variables: 
%%% mode: latex
%%% TeX-engine:xetex
%%% TeX-PDF-mode: t
%%% End:
