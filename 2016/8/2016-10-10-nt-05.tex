\input{../../main}

\setphysstyle{ГЦФО 8}{Серия НТ-05}{10.10.2016}

\begin{document}

\Large

\task{ Бассейн объемом $V$ и глубиной $h$ разделен вертикальной
  перегородкой на две равные части. В одной половине бассейна
  находится вода с плотностью $\rho_1$ , а в другой --- масло с
  плотностью $\rho_2 < \rho_1$ . Какое количество теплоты выделится,
  если перегородку убрать?}

\task{ Альпинистская верёвка подчиняется закону Гука, пока не
  разрывается при силе натяжения $T = 22000\mbox{ Н}$, будучи
  растянутой на $\alpha = 25\%$ от своей первоначальной длины. Во
  время испытания один конец верёвки длиной $L$ закрепляют на стене, а
  свободный конец поднимают вертикально вверх. К свободному концу
  привязывают груз массой $m$ и отпускают его с нулевой начальной
  скоростью. При какой максимальной массе груза $m$ верёвка обязана
  выдержать рывок?}

\task{ Насос перекачивает воду в большую емкость, расположенную на
  высоте $H = 1\mbox{ м}$ по трубе диаметром $d = 10\mbox{ см}$. Какая
  часть энергии насоса расходуется на поднятие воды, если насос
  перекачивает $\mu = 100$ литров в секунду? }


\end{document}

%%% Local Variables: 
%%% mode: latex
%%% TeX-engine:xetex
%%% TeX-PDF-mode: t
%%% End:
