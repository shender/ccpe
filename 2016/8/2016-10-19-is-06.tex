\input{../../main}

\begin{document}

\setphysstyle{ГЦФО 8}{Серия Ш-06}{19.10.2016}

\large
\setcounter{notask}{18}

\taskpic{ Плотность масла измеряют в опыте, схема которого показана на
  рисунке. Сосуд разделён на две части вертикальной перегородкой. В
  одну часть сосуда налита вода, в другую --- масло. В перегородку
  встроен шарнир, который может вращаться без трения. В шарнир
  вставлена однородная сосновая линейка, которая находится в
  равновесии. Длина левой части линейки равна $l_1=40$ см, правой ---
  $l_2=60$ см. Плотность воды равна $\rho_0 = 1000 \kgm$, плотность
  линейки $\rho = 600 \kgm$. Найдите плотность масла. }
{
  \begin{tikzpicture}
    \draw[very thick] (0,2) -- (0,0) -- (3,0) -- (3,2);
    \draw[thick] (0,1.8) -- ++(3,0);
    \draw[very thick] (1.5,0) -- ++(0,2);
    \draw[thick] (0.9,1) rectangle ++(1.4,-0.1);
    \draw[fill=white] (1.5,0.95) circle (0.13cm);
    \draw (1.25,1) node[above,blue] {\small $l_1$};
    \draw (1.9,1) node[above,blue] {\small $l_2$};
    \draw (0.5,0.2) node {\small вода};
    \draw (2.3,0.2) node {\small масло}; 
  \end{tikzpicture}
}
% МФО-2007

\taskpic[5.5cm]{ Ко дну сосуда при помощи шарнира прикреплена за конец тонкая
  однородная палочка длиной $L$. В сосуд медленно наливают воду и
  отмечают, какая часть длины палочки $L_{\text{п}}$ оказывается под
  водой. График зависимости $L_{\text{п}}/L$ от высоты $h$ уровня
  жидкости над дном сосуда приведён на рисунке. Определите плотность
  материала палочки. Плотность воды равна $\rho_0$. }
{
  \begin{tikzpicture}
    \draw[thick,->] (0,0) node[left] {$0$} -- (4,0) node[right] {$h$};
    \draw[thick,->] (0,0) -- (0,4) node[right] {$L_{\text{п}}/L$} ;
    \coordinate (A) at (1,1);
    \coordinate (B) at (0,1);
    \coordinate (C) at ($(A)+(45:2.5cm)$);
    \draw[very thick, red] (B) node[left,blue] {$\lambda$} -- (A) --
    (C) -- ++(1,0);
    \draw[dashed,thick] (0,0) -- (A) (C) -- ($(0,0)!(C)!(4,0)$)
    node[below,blue] {$L$};
    \draw[dashed,thick] (C) -- ($(0,0)!(C)!(0,4)$) node[left,blue]
    {$1$};
  \end{tikzpicture}
}
% МФО-2006

\taskpic{ Цилиндр радиуса $r$, лежащий на подставке, разрезан пополам
  по вертикальной плоскости, проходящей через его ось. Масса каждой
  половины цилиндра равна $m$, а их центры тяжести находятся на
  расстоянии $l$ от оси цилиндра. Чтобы цилиндр не распался, через
  него перекинули невесомую нерастяжимую нить с одинаковыми грузами на
  концах. Найти минимальную массу грузов, не допускающих распада
  цилиндра. Трением пренебречь. }
{
  \begin{tikzpicture}
    \draw[thick, interface] (2,0) -- (0,0);
    \draw[thick] (1,1.2) circle (1.2cm);
    \draw (1,0) -- ++(0,2.4);
    \draw[thick] (-0.2,1.2) -- ++(0,-2);
    \draw[thick] (2.2,1.2) -- ++(0,-2);
    \draw[thick] (-0.4,-0.8) rectangle ++(0.4,-1);
    \draw[thick] (2,-0.8) rectangle ++(0.4,-1);
  \end{tikzpicture}
}
% НГУ-1, 1.111

\end{document}


%%% Local Variables: 
%%% mode: latex
%%% TeX-engine:xetex
%%% TeX-PDF-mode: t
%%% End:
