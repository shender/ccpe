\documentclass[a4paper, 12pt]{article}
% math symbols
\usepackage{amssymb}
\usepackage{amsmath}
\usepackage{mathrsfs}
\usepackage{physsummer}


\usepackage{enumitem}
\usepackage[margin = 2cm]{geometry}

\tolerance = 1000
\emergencystretch = 0.74cm



\pagestyle{empty}
\parindent = 0mm

\begin{document}

\setphysstyle{ГЦФО 8}{Серия Ш-10}{23.11.2016}

\setcounter{notask}{29}

\Large

\task{ Кусок охлаждённого льда поместили в калориметр. В таблице
  приведены результаты измерений температуры содержимого калориметра в
  зависимости от времени. Постройте график зависимости температуры
  льда и воды от времени. На основании экспериментальных данных
  определите удельные теплоёмкости $c_{\mbox{л}}$ льда и воды
  $c_{\mbox{в}}$. Удельная теплота плавления льда $\lambda=330$
  кДж/кг. Теплоёмкостью калориметра пренебречь, подводимую тепловую
  мощность считайте постоянной.  
\begin{center}
  \begin{tabular}{|c|c|c|c|c|c|c|c|c|c|c|}
    \hline
    $\tau$, с & 0 & 5 & 10 & 15 & 20 & 320 & 330 & 340 & 350 & 360\\
    \hline
    $t$, $^{\circ}$C & -4,8 & -2,5 & 0,0 & 0,0 & 0,0 & 0,0 & 0,0 & 0,0 & 2,5 & 4,9\\
    \hline
  \end{tabular}
\end{center}}
% 2009-09, муниципальный этап

\task{ В калориметр, содержащий $m_{\text{в}}=1{,}5$ кг воды при
  температуре $t_{\text{в}}=20^{\circ}$C, положили $m_{\text{л}}=10$
  кг льда, имеющего температуру $t_{\text{л}}=-10^{\circ}$C. Какая
  температура $T$ установится в калориметре? }
% ГГК, 11.3

\task{ В калориметр, содержащий лёд массой $m_{\text{л}}=100$ г при
  температуре $t_{\text{л}}=0^{\circ}$C, впускают пар при температуре
  $t_{\text{п}}=100^{\circ}$C. Сколько воды оказалось в калориметре,
  когда весь лёд растаял? Температура образовавшейся воды равна
  $0^{\circ}$C. }
% ГГК, 11.7

\end{document}

%%% Local Variables: 
%%% mode: latex
%%% TeX-engine:xetex
%%% TeX-PDF-mode: t
%%% End:
