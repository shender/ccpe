\input{../../main}

\begin{document}

\setphysstyle{ГЦФО 8}{Серия НТ-10}{21.11.2016}

\large

\task{В калориметре находился лёд массой
  $m_{\mbox{л}} = 0,5 \mbox{ кг}$ при температуре
  $t_{\mbox{л}} = - 20^\circ$C. Удельная теплоёмкость льда
  $c_{\mbox{л}} = 2100 \mbox{ Дж/кг}^\circ$C, а его удельная теплота
  плавления $\lambda = 340\mbox{ кДж/кг}$. В калориметр впустили пар
  массой $m_{\mbox{п}} = 60 \mbox{ г}$ при температуре
  $t_{\mbox{п}} = 100^\circ$C. Какая температура установится в
  калориметре? Удельная теплоёмкость воды
  $c_{\mbox{в}} = 4200 \mbox{ Дж/кг}^\circ$C, удельная теплота
  парообразования воды $L = 2,2 \cdot 10^6\mbox{ Дж/кг}$.\\
  Теплоёмкостью калориметра и потерями тепла пренебречь.}

\task{В калориметр, содержащий $100 \mbox{ г}$ воды с начальной
  температурой $12^\circ$C, поместили тело с начальной температурой
  $27^\circ$C. Из справочника известно, что удельная теплоёмкость
  тела в 4 раза меньше, чем у воды. Используя график зависимости
  температур воды и тела от времени, найдите массу тела. \\
  Теплоёмкостью калориметра пренебречь.}

\task{Два литра воды нагревают на электроплитке мощностью
  $W = 500\mbox{ Вт}$. При этом часть тепла теряется теряется в
  окружающую среду, график зависимости мощности теплопотерь от времени
  показан на рисунке. Начальная температура воды равна $20^\circ
  $C. За какое время вода нагреется до $30^\circ$C?}
\ \\

\begin{tikzpicture}[scale = 0.55]
	\draw[->] (-0,0) -- (10.5, 0) node[right] {$t, \mbox{мин}$};
    \draw[->] (0,-0) -- (0, 8.5) node[above] {$T, \mbox{}^\circ C$};
    \foreach \x/\xtext in {0, 1, ..., 9} {
     	\draw[color = gray, xshift=\x cm] (1, 0) -- (1, 8);
     }
     \foreach \y/\ytext in {0, 1, ..., 7} {
     	\draw[color = gray, yshift=\y cm] (0, 1) -- (10, 1);
     }
      \draw[very thick] (0, 2) -- (10, 3.2);
      \draw[very thick] (0, 7) -- (10, 4.6);	
      \draw (-0.65, 1) node{9};
      \draw (-0.65, 3) node{15};
      \draw (-0.65, 5) node{21};
      \draw (-0.65, 7) node{27};
      \draw (2, -0.6) node{2};
      \draw (4, -0.6) node{4};
      \draw (6, -0.6) node{6};
      \draw (8, -0.6) node{8};
      \draw (10, -0.6) node{10};
\end{tikzpicture}
\begin{tikzpicture}[scale = 0.7]
  \draw[->] (0,0) -- (8.5, 0) node[right] {$t, \mbox{cек}$};
  \draw[->] (0,-0) -- (0, 6.5) node[above] {$P, \mbox{Вт}$};
  \draw (0, 2) -- (8, 2);
  \draw (0, 4) -- (8, 4);
  \draw (0, 6) -- (8, 6);
  \draw (4, 0) -- (4, 6);
  \draw (8, 0) -- (8, 6);
  \draw[very thick] (0, 2) -- (8, 6);
  \draw (-0.4, -0.5) node{0};
  \draw (-0.75, 2) node{100};
  \draw (-0.75, 4) node{200};
  \draw (-0.75, 6) node{300};
  \draw (4, -0.6) node{200};
  \draw (8, -0.6) node{400};
\end{tikzpicture}


\end{document}


%%% Local Variables: 
%%% mode: latex
%%% TeX-engine:xetex
%%% TeX-PDF-mode: t
%%% End:
