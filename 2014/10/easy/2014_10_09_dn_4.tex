\input{../../input/main}

\usetikzlibrary{hobby}

\begin{document}

\begin{center}
  \Large{\textbf{Городской центр физического образования, 10 класс.}\\
  \textit{Серия 4, 9 октября 2014.}}
\end{center}

\begin{center}
  \Large \textbf{Для разогрева.}
\end{center}

\Large

\task{ Какое давление имеет 1 кг азота в объёме $1 \mbox{ м}^3$ при
  температуре $27^{\circ}\,C$? Атомный вес азота 14.  }

\begin{center}
  \Large \textbf{ Идеальный газ.}
\end{center}

\task{ Газофазная эпитаксия --- это метод получения тонких пленок
  путем осаждения молекул из газа. Для этого подложку помещают в
  идеальный газ с концентрацией $n$ и температурой $T$. Считая, что
  все молекулы газа, попадающие на подложку, встраиваются в
  эпитаксиальную пленку, найдите, какое давление газ оказывает на
  подложку. }

\task{ Баллон вместимостью 50 л наполнили воздухом при температуре
  $27^{\circ}\,C$ до давления 10 МПа. Какой объём воды можно вытеснить
  из цистерны подводной лодки воздухом из этого баллона, если
  вытеснение производить на глубине 40 м? Температура после расширения
  равна $3^{\circ}\,C$. }

\task{ Предположим, что планету массой $M$ и радиуса $r$ окружает
  атмосфера постоянной плотности, состоящая из газа с молярной массой
  $\mu$. Определите температуру $T$ атмосферы на поверхности планеты,
  если толщина атмосферы равна $h$ ($h \ll r$).  }

\end{document}


%%% Local Variables: 
%%% mode: latex
%%% TeX-engine:xetex
%%% TeX-PDF-mode: t
%%% End:
