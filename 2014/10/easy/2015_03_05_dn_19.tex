\input{../../input/main}

\begin{document}

\begin{center}
  \Large{\textbf{Городской центр физического образования, 10 класс.}\\
  \textit{Серия 19, 5 марта 2015.}}
\end{center}

\begin{center}
  \Large\textbf{Вперёд к конденсаторам. }
\end{center}

\Large

\task{ Три проводящие концентрические сферы радиусов $r_0$, $2r_0$ и
  $3r_0$ имеют заряд $q$, $2q$, $-3q$ соответственно. Определите
  потенциал каждой из сфер и постройте график зависимости $\varphi(r)$. }

\task{ Система состоит из двух концентрических проводящих сфер ---
  внутренней радиуса $R_1$ и внешней радиуса $R_2$. Внутренняя сфера
  имеет заряд $q$, внешняя --- заземлена. Найдите напряженность и
  потенциал электрического поля в зависимости от расстояния до центра
  сфер. }

\task{ Сферический конденсатор состоит из двух концентрических сфер
  радиусами $r_1=5$ см и $r_2=5{,}5$ см. Пространство между обкладками
  конденсатора заполнено маслом ($\vareps=2{,}2$). Определите: 1)
  ёмкость этого конденсатора; 2) шар какого радиуса, помещенный в
  масло, обладает такой же ёмкостью.  }

\task{ Металлический заряженный шар радиуса $R$ окружен толстым
  сферическим слоем диэлектрика ($\vareps=2$) радиуса $3R$. Нарисуйте
  картину силовых линий поля. Почему поле скачкообразно изменяется при
  переходе через границу диэлектрика?  }

\end{document}


%%% Local Variables: 
%%% mode: latex
%%% TeX-engine:xetex
%%% TeX-PDF-mode: t
%%% End:
