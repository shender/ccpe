\input{../../input/main}

\usetikzlibrary{hobby}

\begin{document}

\begin{center}
  \Large{\textbf{Городской центр физического образования, 10 класс.}\\
  \textit{Серия 6, 23 октября 2014.}}
\end{center}

\begin{center}
  \Large \textbf{ Нагреваем газы.}
\end{center}

\large

\task{ Имеются два теплоизолированных сосуда равного объёма,
  наполненные одинаковым газом. Газам в сосудах сообщают одинаковое
  количество теплоты. Начальные температуры газов равны. Начальное
  давление газа во втором сосуде в 2 раза больше, чем в
  первом. Конечная температура газа в первом сосуде $T_1=400$K, во
  втором $T_2=300$K. Найдите начальную температуру газов. }

\task{ Горизонтально расположенный теплоизолированный сосуд разделен
  на три части $V_1,V_2,V_3$ закрепленными поршнями, между которыми
  находятся различные массы идеального одноатомного газа при различных
  начальных температурах и давлениях $p_1,p_2,p_3$. Определить
  давление в каждой секции сосуда после того, как поршни получили
  возможность свободно перемещаться, а в сосуде установилось
  термодинамическое равновесие. Теплоемкостью поршней пренебречь. }

\task{ В цилиндре под поршнем находится при нормальных условиях порция
  гелия в количестве $\nu=2$ моль. Ей сообщают количество теплоты
  $Q=100$ Дж, при этом температура гелия увеличивается на $\Delta
  T=10$K. Оцените изменение объёма газа в этом процессе, считая его
  теплоёмкость постоянной.  }

\taskpic{ На столе стоит вертикальный теплоизолированный
  цилиндрический сосуд. В него вставлен поршень и неподвижная
  перегородка. Поршень тяжёлый, теплонепроницаемый и может двигаться в
  цилиндре без трения. Перегородка — лёгкая и теплопроводящая. В
  каждой из частей сосуда находится по $\nu$ молей идеального
  одноатомного газа. Вначале система находилась в тепловом равновесии,
  а обе части сосуда имели высоту $L$. Потом систему медленно нагрели,
  сообщив ей количество теплоты $\Delta Q$. На какую величину $\Delta
  T$ изменилась температура газов? Какова теплоёмкость $C$ системы в
  этом процессе? Теплоёмкостью стенок сосуда, поршня и перегородки
  пренебречь. }
{
  \begin{tikzpicture}
    \draw[fill=gray!20] (0,2.7) rectangle (3.8,2.9);
    \draw[pattern=north east lines] (0,5.5) rectangle (3.8,5.8);
    \draw[very thick] (0,6) -- (0,0) -- (3.8,0) -- (3.8,6);
    \draw[blue,thick,<->] (1,0) -- (1,2.7) node[midway,blue,fill=white] {$L$}; 
    \draw[blue,thick,<->] (1,2.9) -- (1,5.5)
    node[midway,blue,fill=white] {$L$};
    \draw (1.9,1.5) node[blue] {$\nu$};
    \draw (1.9,4.4) node[blue] {$\nu$}; 
  \end{tikzpicture}
}

\end{document}


%%% Local Variables: 
%%% mode: latex
%%% TeX-engine:xetex
%%% TeX-PDF-mode: t
%%% End:
