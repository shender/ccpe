\documentclass[a4paper, 12pt]{article}
% math symbols
\usepackage{amssymb}
\usepackage{amsmath}
\usepackage{mathrsfs}
\usepackage{physsummer}


\usepackage{enumitem}
\usepackage[margin = 2cm]{geometry}

\tolerance = 1000
\emergencystretch = 0.74cm



\pagestyle{empty}
\parindent = 0mm

\begin{document}

\begin{center}
  \Large{\textbf{Городской центр физического образования, 10 класс.}\\
  \textit{Серия 18${}^{\ast}$, 26 февраля 2015.}}
\end{center}

\begin{center}
  \large\textbf{ Электростатика. }
\end{center}

\large

\task{ Если по квадратной диэлектрической пластине равномерно
  распределить заряд $q$, то потенциал в ее центре будет равен
  $\varphi_1$. Если из шести таких пластин с зарядом $q$ на каждой
  составить полый куб, то потенциал в его центре будет равен
  $\varphi_2$. Определите потенциал в вершине такого куба. Потенциал на
  бесконечности примите равным нулю. }

\task{ Равномерно заряженная тонкая нить длины $L$ имеет заряд
  $Q$. Нить положили неподвижно на скользкий ровный стол, при этом ее
  натяжение в середине оказалось равно $T_0$. Затем нить продели в две
  маленькие бусинки с зарядом $q$ каждая. Бусинки прикрепили к столу
  на расстоянии $l$ друг от друга, при этом продетая в них нить
  расположилась на столе прямолинейно, симметрично относительно
  бусинок. Сила трения между нитью и бусинками отсутствует. Чему
  теперь окажется равным натяжение нити в середине? Нить нерастяжима,
  заряды нити и бусинок одноименны. Размером бусинок пренебречь, нить
  невесома.  }

\taskpic{ Маленький шарик и тонкий непроводящий стержень длиной $L$,
  массы которых $m$ одинаковы, подвешены к потолку на нитях одинаковой
  большой длины $R \gg L$. Нити позволяют шарику и стержню двигаться
  только в одной вертикальной плоскости. Сначала шарик и стержень не
  были заряжены и висели так, что почти соприкасались друг с другом,
  причем шарик находился возле одного из концов стержня. Шарику и
  стержню сообщили одинаковые электрические заряды $Q$, причем заряд
  на стержне распределили равномерно по его длине. На каком расстоянии
  $x$ окажутся в положении равновесия шарик и тот конец стержня, возле
  которого шарик сначала находился? Считайте, что диаметр шарика много
  меньше $x$, а $x$ много меньше длины стержня.  }
{
  \begin{tikzpicture}
    \draw[thick,interface] (0,0) -- (3.5,0);
    \draw[thick] (0.5,0) --++(down:3.1cm);
    \draw[fill=black] (0.5,-3.1) circle (0.08cm);
    \draw[thick] (1.2,0) -- ++(down:3cm);
    \draw[thick] (3.2,0) -- ++(down:3cm);
    \draw[black,fill=gray] (0.7,-3) rectangle (3.4,-3.2); 
  \end{tikzpicture}
}

\end{document}


%%% Local Variables: 
%%% mode: latex
%%% TeX-engine:xetex
%%% TeX-PDF-mode: t
%%% End:
