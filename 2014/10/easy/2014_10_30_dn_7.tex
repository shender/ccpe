\documentclass[a4paper, 12pt]{article}
% math symbols
\usepackage{amssymb}
\usepackage{amsmath}
\usepackage{mathrsfs}
\usepackage{physsummer}


\usepackage{enumitem}
\usepackage[margin = 2cm]{geometry}

\tolerance = 1000
\emergencystretch = 0.74cm



\pagestyle{empty}
\parindent = 0mm

\usetikzlibrary{hobby}

\begin{document}

\begin{center}
  \Large{\textbf{Городской центр физического образования, 10 класс.}\\
  \textit{Серия 7, 30 октября 2014.}}
\end{center}

\begin{center}
  \Large \textbf{ Газы и кое-что ещё. }
\end{center}

\Large

\task{ Поршень массы $M$, закрывающий объём $V_0$ одноатомного газа
  при давлении $P_0$ и температуре $T_0$, движется со скоростью
  $u$. Определите температуру и объём газа при максимальном
  сжатии. Система теплоизолирована, теплоёмкостями поршня и сосуда
  пренебречь. }

\task{ В сосуде под лёгким поршнем находится гелий. Высота поршня над
  дном сосуда $H$. Сосуд и поршень теплоизолирующие, поршень может
  двигаться без трения. С некоторой высоты на поршень падает без
  начальной скорости маленький упругий шарик. Какова должна быть эта
  высота, чтобы после установления в системе равновесия (шарик лежит
  на поршне) положение поршня не изменилось? }

\task{ Горизонтальный цилиндр, заполненный идеальным газом, закрыт
  невесомым поршнем площади $S$. Внутри цилиндра находится спираль
  сопротивления $r$, по которой течет ток $I$. Поршень равномерно
  движется со скоростью $V$. Определите теплоёмкость одного моля газа
  в этом процессе. Атмосферное давление равно $p$. }

\task{ Внутри гладкой горизонтальной трубы находятся два
  легкоподвижных поршня, соединённых между собой упругой
  пружиной. Между поршнями находится один моль идеального одноатомного
  газа при температуре $T_0 = 300$K. Газ нагрели до температуры
  $T_1=400$K. Какое количество теплоты было сообщено газу при
  нагревании, если длина пружины увеличилась в $h=1.1$ раза?  }
\begin{center}
  \begin{tikzpicture}
    \draw[very thick] (0,0) -- (6,0);
    \draw[very thick] (0,2) -- (6,2);
    \draw[thick,pattern=north east lines] (1,0) rectangle (1.2,2);
    \draw[thick,pattern=north east lines] (4.8,0) rectangle (5,2);
    \draw[spring] (1.2,1) -- (4.8,1); 
  \end{tikzpicture}
\end{center}

\end{document}


%%% Local Variables: 
%%% mode: latex
%%% TeX-engine:xetex
%%% TeX-PDF-mode: t
%%% End:
