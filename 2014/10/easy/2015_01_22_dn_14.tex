\input{../../input/main}

\begin{document}

\begin{center}
  \Large{\textbf{Городской центр физического образования, 10 класс.}\\
  \textit{Серия 14, 22 января 2015.}}
\end{center}

\begin{center}
  \Large\textbf{ Распределение по скоростям. }
\end{center}

\Large

\task{ В сосуде с газом поддерживается постоянная температура $T_0$. Вне
  полости находится такой же газ давление которого $P$, и температура
  $T$. Чему равно давление газа в полости, если в её стенке имеется
  небольшое отверстие? Газ считать разреженным. }
% Паршаков 4.2.5

\task{ Оцените, во сколько раз поток газа, вытекающего из сосуда через
  цилиндрический канал радиуса $R$ и длины $L$, меньше потока газа,
  вытекающего через отверстие радиуса $R$. Считать, что стенки канала
  поглощают молекулы. }
% Савченко 5.1.10

\task{ Скорости частиц, движущихся в потоке, имеют одно направление и
  лежат в интервале от $v_0$ до $2v_0$. График функции распределения
  частиц по скоростям имеет вид прямоугольника. Чему равно значение
  функции распределения? Как изменяется функция распределения, если на
  частицы в течение времени $\tau$ вдоль их скорости действует сила
  $F$? Масса каждой частицы равна $m$. }
% Савченко 5.2.12

\task{ Найдите отношение числа молекул водорода, имеющих проекцию
  скорости на ось $x$ в интервале от $3000$ до $3010$ м/с, на ось
  $y$~---~ в интервале от $3000$ до $3010$ м/с, на ось $z$~---~в
  интервале от $3000$ до $3002$ м/с, к числу молекул водорода, имеющих
  проекцию скорости на ось $x$ в интервале от $1500$ до $1505$ м/с, на
  ось $y$~---~ в интервале от $1500$ до $1501$ м/с, на ось $z$~---~в
  интервале от $1500$ до $1502$ м/с. Температура водорода 300К. }
% Савченко 5.2.5

\end{document}


%%% Local Variables: 
%%% mode: latex
%%% TeX-engine:xetex
%%% TeX-PDF-mode: t
%%% End:
