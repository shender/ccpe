\documentclass[a4paper, 12pt]{article}
% math symbols
\usepackage{amssymb}
\usepackage{amsmath}
\usepackage{mathrsfs}
\usepackage{physsummer}


\usepackage{enumitem}
\usepackage[margin = 2cm]{geometry}

\tolerance = 1000
\emergencystretch = 0.74cm



\pagestyle{empty}
\parindent = 0mm

\begin{document}

\begin{center}
  \Large{\textbf{Городской центр физического образования, 10 класс.}\\
  \textit{Серия 22Ш, 4 мая 2015.}}
\end{center}

\begin{center}
  \Large\textbf{ Повторение --- мать учения. }
\end{center}

\Large

\task{ При каком коэффициенте трения человек сможет вбежать на горку
  высотой $h=10$ м с углом наклона $\alpha =0{,}1$ рад за время $t=1$
  с без предварительного разгона? Считать, что мощность человека не
  ограничивает время движения, а сопротивление воздуха мало. }

\task{ В сосуде объёмом $V_1=20$ л находится вода, насыщенный водяной
  пар и воздух. Объём сосуда при постоянной температуре медленно
  увеличивают до $V_2=40$ л, давление в сосуде при этом уменьшается от
  $p_1=3$ атм до $p_2=2$ атм. Определите массу воды в сосуде в конце
  опыта, если общая масса воды и пара составляет $m=36$ г. Объёмом,
  занимаемым жидкостью, в обоих случаях пренебречь. }

\task{ Три конденсатора с ёмкостями $C_1=C_{0}$, $C_2=2C_0$,
  $C_3=3C_0$, каждый из которых заряжен от батареи с ЭДС
  $\mathcal{E}$, и резистор с сопротивлением $R$ включены в схему,
  изображённую на рисунке. 1) Чему будет равен ток в цепи сразу после
  замыкания ключа? 2) Какие разности потенциалов установятся на
  конденсаторах после нового равновесного состояния? 3) Какое
  количество теплоты выделится в резисторе после замыкания ключа? }
\begin{center}
  \begin{tikzpicture}[circuit ee IEC, thick]
    \draw (0,0) to [capacitor={info={$C_1$},info=150:{$+$},info=30:{$-$}}] (2,0) to
    [capacitor={info={$C_2$},info=150:{$+$},info=30:{$-$}}] (4,0) to[make 
    contact] ++(down:2cm); 
    \draw (0,0) to [resistor={info={$R$}}] ++(down:2cm)
    to[capacitor={info={$C_3$},info=150:{$+$},info=30:{$-$}}] ++ (right:4cm);  
  \end{tikzpicture}
\end{center}


\end{document}


%%% Local Variables: 
%%% mode: latex
%%% TeX-engine:xetex
%%% TeX-PDF-mode: t
%%% End:
