\input{../../input/main}

\begin{document}

\begin{center}
  \Large{\textbf{Городской центр физического образования, 10 класс.}\\
  \textit{Серия 1Ш, 15 сентября 2014.}}
\end{center}
\large

\begin{center}
  \Large \textbf{Для разгона.}
\end{center}

\Large

\task{ Два камня падают в шахту. Второй камень начал свое падение на 1
  с позже первого. Определить движение первого камня относительно
  второго. Ускорение свободного падения $g = 9.8 \mbox{ м/с}^{2}$.}

% Шаскольская - Эльцин
% Московская олимпиада, 1946

\begin{center}
  \Large \textbf{Всё о трении, \\ или \\ как важно рисовать картинки.}
\end{center}

\task{ Под каким углом $\alpha$ нужно тянуть за верёвку тяжёлый ящик,
  чтобы с наименьшим усилием передвигать его волоком по горизонтальной
  поверхности? Коэффициент трения между ящиком и поверхностью равен
  $\mu$. }

\task{ То же, что в предыдущей задаче, но теперь требуется найти
  наименьшую силу (и угол её приложения), чтобы сообщать ящику
  ускорение $a$ по горизонтальной оси. }

\task{ Под каким углом $\alpha$ нужно тянуть за верёвку тяжёлый ящик,
  чтобы с наименьшим усилием передвигать его волоком по наклонной
  плоскости с углом $\beta$ в основании? Коэффициент трения между
  ящиком и плоскостью равен $\mu$.  }

% \task{ На наклонной плоскости, составляющей угол $\alpha$ с
%   горизонтом, лежат две доски, одна на другой. Можно ли подобрать
%   такие значения масс досок $m_1$ и $m_2$, коэффициентов трения досок
%   о плоскость $\mu_1$ и друг о друга $\mu_2$, чтобы нижняя доска
%   выскользнула из-под верхней? В начальный момент доски покоятся.}

\task{ Однородной шайбе, лежащей на горизонтальной шероховатой
  поверхности, сообщают вращение и одновременно поступательное
  движение со скоростью $v_0$. По какой траектории движется центр
  шайбы? В каком случае шайба пройдёт б\textbf{о}льший путь до остановки:
  вращаясь или не вращаясь?}

\end{document}


%%% Local Variables: 
%%% mode: latex
%%% TeX-engine:xetex
%%% TeX-PDF-mode: t
%%% End:
