\input{../../input/main}

\begin{document}

\begin{center}
  \Large{\textbf{Городской центр физического образования, 10 класс.}\\
  \textit{Серия 13Ш, 26 января 2015.}}
\end{center}

\begin{center}
  \Large\textbf{ ВНЕЗАПНО оптика. }
\end{center}

\Large

\task{ В комнате длины $L$ и высоты $H$ висит на стене плоское
  зеркало. Человек смотрит в него, находясь на расстоянии $l$ от той
  стены, на которой оно висит. Какова должна быть наименьшая высота
  зеркала, чтобы человек мог видеть стену, находящуюся за его спиной,
  во всю высоту? }
% НГУ-1, 4.1

\task{ Вогнутое сферическое металлическое зеркало, направленное на
  Солнце, сфокусировало свет в точку, расположенную на оси зеркала на
  расстоянии $L_1$ от его центра. Температура зеркала была при этом
  равна $t_1$. На каком расстоянии от центра зеркала будет находиться
  изображение после того, как зеркало нагреется до температуры $t_2$?
  Полюс зеркала закреплён. Температурный коэффициент линейного
  расширения равен $\alpha$. }
% НГУ-1, 4.4

\task{ Вдоль главной оптической оси собирающей линзы с фокусным
  расстоянием $F=5$ см движутся навстречу друг другу два светлячка,
  находящихся по разные стороны линзы. Скорость светлячков одна и та
  же~---~$v=2$ см/с. Через какое время первый светлячок встретится с
  изображением второго, если в начальный момент они находились на
  расстояниях $l_1=20$ см и $l_2=15$ см от линзы? }
% НГУ-1, 4.24

\end{document}


%%% Local Variables: 
%%% mode: latex
%%% TeX-engine:xetex
%%% TeX-PDF-mode: t
%%% End:
