\input{../../input/main}

\begin{document}

\begin{center}
  \Large{\textbf{Городской центр физического образования, 10 класс.}\\
  \textit{Серия 12Ш, 12 января 2015.}}
\end{center}

\begin{center}
  \Large\textbf{ Несложная смесь. }
\end{center}

\Large

\task{ Тело массой $m_1$, расположенное на горизонтальной плоскости,
  тянут за привязанную к нему верёвку, действуя с силой $F_1$,
  направленной параллельно плоскости. Масса верёвки $m_2$, длина
  $l$. Найдите натяжение верёвки в зависимости от расстояния до
  тела. Верёвка нерастяжима, трения нет. }
% Манида, 45

\task{ Имеется проволока квадратного сечения. Для того, чтобы её
  расплавить, по ней надо пропустить ток $I=10$ А. Какой ток надо
  пропустить, чтобы расплавить проволоку круглого сечения с той же
  площадью сечения? Мощность теплопотерь в окружающую среду $P$
  подчиняется закону $P=kS (T-T_0)$, где $S$~---~площадь поверхности
  проволоки, $T$~---~её температура, $T_0$~---~температура окружающей
  среды на большом расстоянии от проволоки, $k$~---~коэффициент,
  одинаковый в обоих случаях. }
% Манида, 293

\task{ В кастрюле находится вода при температуре
  $60^{\circ}$C. Кастрюлю закрывают крышкой, масса которой $m=5$ кг,
  площадь $S=100\mbox{ см}^{2}$. Кастрюлю медленно нагревают до
  $70^{\circ}$C. Сколько раз подпрыгнет крышка за это время? Давление
  насыщенных паров при $60^{\circ}$C равно $p_1=2 \cdot 10^4$ Па, при
  $70^{\circ}$C $p_2=3{,}1 \cdot 10^4$ Па, атмосферное давление
  $p_0=10^5$ Па. Считайте, что при подпрыгивании крышки давление в
  кастрюле падает до атмосферного. }
% Манида, 230

\end{document}


%%% Local Variables: 
%%% mode: latex
%%% TeX-engine:xetex
%%% TeX-PDF-mode: t
%%% End:
