\input{../../input/main}

\begin{document}

\begin{center}
  \Large{\textbf{Городской центр физического образования, 10 класс.}\\
  \textit{Серия 14Ш, 2 февраля 2015.}}
\end{center}

\begin{center}
  \Large\textbf{ Продолжим оптику. }
\end{center}

\Large

\task{ Собирательная линза с фокусным расстоянием $f$ имеет две
  поверхности~---~одну плоскую, вторую выпуклую. Плоскую поверхность
  посеребрили так, что она стала полностью отражать свет. Как в общем
  случае построить изображение в этой оптической системе? }
% Манида, 333

\task{ Цилиндрическая часть стеклянной бутылки имеет внутренний радиус
  $r$ и внешний радиус $R$. Бутылку заполняют молоком. Показатель
  преломления стекла равен $n_1$, молока $n_{2}$, причем $n_2>n_1$. При
  каком соотношении между $R$ и $r$ при взгляде сбоку будет казаться, что
  толщина стенок бутылки равна нулю? }
% Город-2006, 10 класс

\taskpic{ Маленькая муха летает по кругу радиусом $r=1$ см с
  постоянной скоростью $u=10$ см/с. Центр круга находится на
  оптической оси собирающей линзы на расстоянии $L=30$ см от
  неё. Фокусное расстояние линзы равно $f=10$ см. Найдите минимальную
  и максимальную величины скорости изображения мухи, создаваемого
  линзой. При каком положении изображения они достигаются? }
{
  \begin{tikzpicture}
    \draw[very thick,<->] (2,0) -- (2,3);
    \draw[dashed,thick] (-1,1.5) -- (2.8,1.5);
    \draw[thick] (0,1.5) circle (0.3cm); 
  \end{tikzpicture}
}
% Город-2005, 10 класс

\end{document}


%%% Local Variables: 
%%% mode: latex
%%% TeX-engine:xetex
%%% TeX-PDF-mode: t
%%% End:
