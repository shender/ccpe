\input{../../input/main}

\begin{document}

\begin{center}
  \Large{\textbf{Городской центр физического образования, 10 класс.}\\
  \textit{Серия 21Ш, 27 апреля 2015.}}
\end{center}

\begin{center}
  \Large\textbf{ Непростое магнитное поле. }
\end{center}

\Large

\task{ Определить индукцию магнитного поля в центре однородной
  металлической пластинки, имеющей форму равностороннего треугольника
  со стороной $l$, если ток $I$ подводится по проводам, присоединённым
  к двум вершинам треугольника. Магнитным полем подводящих проводов
  пренебречь. }

\task{ Заряженная частица попадает в среду, где на неё действует сила
  сопротивления, пропорциональная скорости. До полной остановки
  частица проходит путь $S=10$ см. Если в среде имеется магнитное
  поле, перпендикулярное скорости частицы, то она при той же начальной
  скорости остановится на расстоянии $l_1=6$ см от точки входа в
  среду. На каком расстоянии $l_2$ от точки входа в среду остановилась
  бы частица, если бы поле было в два раза меньше? }
% БЗК, 3.58

\task{ Две частицы с одинаковыми массами $m$ и зарядами $q$ и $-q$
  начинают с нулевыми начальными скоростями двигаться в однородном
  магнитном поле $B$, перпендикулярном соединяющему их отрезку длины
  $R$. Найдите минимальное значение индукции магнитного поля $B=B_0$
  при котором частицы не столкнутся друг с другом.  }
% Россия, 2004, 11 класс




\end{document}


%%% Local Variables: 
%%% mode: latex
%%% TeX-engine:xetex
%%% TeX-PDF-mode: t
%%% End:
