\input{../../input/main}

\begin{document}

\begin{center}
  \Large{\textbf{Городской центр физического образования, 10 класс.}\\
  \textit{Серия 4Ш, 13 октября 2014.}}
\end{center}

\begin{center}
  \Large\textbf{Виртуальные перемещения.}
\end{center}

\Large

\taskpic{ В коробке \textbf{К} заключен передающий механизм
  неизвестной конструкции. При повороте ручки \textbf{P} вертикальный
  винт \textbf{В} плавно поднимается. При одном полном обороте (радиус
  оборота $r$) винт перемещается вверх на расстояние $h$. На винт
  кладут груз массы $m$. Какое усилие надо приложить к ручке, чтобы
  удержать систему с грузом в равновесии?}  {
  \begin{tikzpicture}
    \draw[thick] (0,0) rectangle (2,2) node[midway] {\textbf{К}};
    \draw[fill=gray] (0.9,2) rectangle (1.1,3) node[midway,left] {\textbf{В}};
    \draw[thick] (0.6,3) rectangle (1.4,3.6) node[midway] {$m$};
    \draw[very thick] (2,1) -- (2.3,1) -- (2.3,0.7) -- (2.6,0.7)
    node[right] {\textbf{Р}}; 
  \end{tikzpicture}
}

\task{ Петля из гибкой тяжёлой цепи массы $m$ надета на глядкий прямой
  круговой конус, высота которого $H$, а радиус основания $R$. Цепь
  покоится в горизонтальной плоскости. Найти натяжение цепи $T$. }

\taskpic{ Два однородных стержня, массы которых $m_1$ и $m_2$,
  опираются на гладкие вертикальные стенки и гладкую горизонтальную
  поверхность. Найдите соотношение между углами $\alpha_1$ и
  $\alpha_2$ при равновесии системы.   }
{
  \begin{tikzpicture}
    \coordinate (a) at (0,2.5);
    \coordinate (b) at (2.1,0);
    \coordinate (c) at (3,2.5);
    \coordinate (d) at (3,0);
    \coordinate (e) at (0,0); 
    \draw[line width=0.6mm] (a) -- (b) node[blue,midway,left=0cm]
    {$m_1$};
    \draw[line width=0.8mm] (b) -- (c) node[blue,midway,above left=-0.1cm]
    {$m_2$};
    \draw[thick,interface] (3,3) -- (3,0);
    \draw[thick,interface] (3,0) -- (0,0);
    \draw[thick,interface] (0,0) -- (0,3);
    \draw (b) ++(30:0.6cm) node[blue] {$\alpha_2$};
    \draw (b) ++(160:0.75cm) node[blue] {$\alpha_1$}; 
  \end{tikzpicture}
}
% Квант, 1980, №9

\end{document}


%%% Local Variables: 
%%% mode: latex
%%% TeX-engine:xetex
%%% TeX-PDF-mode: t
%%% End:
