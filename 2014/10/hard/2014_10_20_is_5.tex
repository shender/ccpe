\input{../../input/main}

\begin{document}

\begin{center}
  \Large{\textbf{Городской центр физического образования, 10 класс.}\\
  \textit{Серия 5Ш, 20 октября 2014.}}
\end{center}

\begin{center}
  \Large\textbf{Газы и молекулы.}
\end{center}

\Large

\task{ Некоторое количество одноатомного газа --- гелия --- занимает
  объём $V=20$ л при давлении $p=0.5$ атм и температуре $T=300$K. Над
  этим газом проводят процесс, при котором ему медленно сообщают
  количество теплоты $Q=40$ Дж. Температура газа при этом
  увеличивается на $\Delta T = 10$K. Сжимается или расширяется газ в
  этом процессе? }
% Зильберман, ШФО, стр. 78

\task{ Найдите среднюю плотность водяного пара над большой лужей при
  температуре $+20^{\circ}$C. Оцените расстояние между соседними
  молекулами пара. Давление насыщенного водяного пара при этой
  температуре составляет 100 Па. }
% Зильберман, ШФО, стр. 78

\task{ В глубинах космоса, вдали от всех других тел неподвижно висит
  длинная пробирка, открытая с одной стороны. Малая часть пробирки у
  её закрытого конца отделена от окружающего пространства тонкой
  перепонкой, между закрытым концом пробирки и перепонкой находится
  небольшое количество гелия. Пробирка с содержимым медленно
  нагревается излучением. Когда её температура достигает 300K,
  перепонка лопается и газ начинает быстро покидать пробирку. Оцените
  скорость пробирки после выхода газа из неё. Масса газа 1 г, масса
  пробирки в 100 раз больше. Теплообменом между газом и пробиркой за
  время выхода газа можно пренебречь. }
% Сорос, 1999

\end{document}


%%% Local Variables: 
%%% mode: latex
%%% TeX-engine:xetex
%%% TeX-PDF-mode: t
%%% End:
