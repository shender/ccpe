\documentclass[a4paper, 12pt]{article}
% math symbols
\usepackage{amssymb}
\usepackage{amsmath}
\usepackage{mathrsfs}
\usepackage{physsummer}


\usepackage{enumitem}
\usepackage[margin = 2cm]{geometry}

\tolerance = 1000
\emergencystretch = 0.74cm



\pagestyle{empty}
\parindent = 0mm

\begin{document}

\begin{center}
  \Large{\textbf{Городской центр физического образования, 10 класс.}\\
  \textit{Серия 10Ш, 8 декабря 2014.}}
\end{center}

\begin{center}
  \Large\textbf{ Термодинамика с особенностями. }
\end{center}

\Large

\task{ В длинном теплоизолированном цилиндре находится идеальный
  одноатомный газ при давлении $p_0$ и температуре $T_0$. Поршень
  соединён со стенкой пружиной, которая в начальном состоянии не
  деформирована. Цилиндр помещён в вакуум. После того, как поршень
  отпустили, газ расширился и занял объём в два раза больше
  первоначального. Равновесие установилось за счёт вязкости
  воздуха. Найдите температуру $T_1$ и давление $p_1$ газа при новом
  положении поршня.  }
\begin{center}
  \begin{tikzpicture}
    \draw[fill=gray!20] (2,0) rectangle (2.3,3); 
    \draw[very thick] (0,0) -- (0,3) -- (6,3);
    \draw[very thick] (0,0) -- (6,0);
    \draw[thick,spring] (0,1.5) -- (2,1.5) node[above,midway]
    {$p_0,T_0$}; 
    \draw (4,1.5) node {вакуум};
  \end{tikzpicture}
\end{center}
% Регион-2011, 10 класс

\task{ Камеру объёмом $V=10$ л при температуре $t_1=0^{\circ}\,C$
  наполнили сухим воздухом, ввели в неё $m=3$ г воды, закрыли, а затем
  нагрели до $t_2=100^{\circ}\,$C. Какое давление установится в
  камере, если первоначальное давление было $p_1=10^5$ Па? }
% Квант, 1992-06, стр. 58

\task{ В цилиндре под поршнем ничтожной массы находится $m_1=10$ г
  насыщенного водяного пара при давлении $p=100$ кПа. В цилиндр
  впрыскивают $m_2=5$ г воды при температуре $t_2=0^{\circ}\,$C. На
  сколько при этом опустится поршень? Площадь сечения поршня $S=100
  \mbox{ см}^2$. Теплоёмкостью цилиндра пренебречь, все необходимые
  физические константы известны. }
% Квант, 1992-06, стр. 56

\end{document}


%%% Local Variables: 
%%% mode: latex
%%% TeX-engine:xetex
%%% TeX-PDF-mode: t
%%% End:
