\input{../../input/main}

\begin{document}

\begin{center}
  \Large{\textbf{Городской центр физического образования, 10 класс.}\\
  \textit{Серия 9Ш, 24 ноября 2014.}}
\end{center}

\begin{center}
  \Large\textbf{ Опять термодинамика. }
\end{center}

\Large

\task{ Цилиндрический сосуд разделен свободно скользящим поршнем на
  две части. Стенки сосуда и поршень тепла не проводят. В одной части
  находятся $\nu_1$ молей гелия, а в другой --- $\nu_2$ молей
  аргона. Поршень быстро вынимают. Найдите установившиеся температуру
  и давление. Объём сосуда $V$, давление на поршень в состоянии
  равновесия равновесия $p$. Газ считать идеальным. При вынимании
  поршня работа не производится. }
% Манида, №221

% \task{ В горизонтальном неподвижном цилиндрическом сосуде, закрытом
%   поршнем массы $M$, находится один моль идеального газа. Газ
%   нагревают; при этом поршень, двигаясь равноускоренно, приобретает
%   скорость $v$. Найдите количество теплоты, сообщённое
%   газу. Теплоёмкостью сосуда и поршня, а также внешним давлением
%   пренебречь. }
% % Квант, 1986-04

\task{  Компрессор, изначально предназначенный для сжатия воздуха,
  используется для сжатия гелия. Обнаружилось, что компрессор
  перегревается. Объясните этот эффект, предполагая, что процесс
  сжатия --- адиабатический, а начальные давления в обоих газах
  равны. }
% Wisconsin, 1014

\task{ Рабочим телом тепловой машины является идеальный одноатомный
  газ. Цикл состоит из изобарного расширения (1,2), адиабатического
  расширения (2,3) и изотермического сжатия (3,1). Модуль работы при
  изотермическом сжатии равен $A_{31}$. Определите, чему может быть
  равна работа газа при адиабатическом расширении $A_{23}$, если у
  указанного цикла КПД $\eta \leq 0.4$? }

\end{document}


%%% Local Variables: 
%%% mode: latex
%%% TeX-engine:xetex
%%% TeX-PDF-mode: t
%%% End:
