\input{../../input/main}

\begin{document}

\begin{center}
  \Large{\textbf{Городской центр физического образования, 10 класс.}\\
  \textit{Серия 19Ш, 6 апреля 2015.}}
\end{center}

\begin{center}
  \Large\textbf{ Конденсаторы в цепях. }
\end{center}

\large

\task{ В схеме, изображённой на рисунке, конденсаторы $C_1$ и $C_2$
  имеют одинаковую ёмкость $C$. В начальный момент ключи \textbf{K1} и
  \textbf{K2} разомкнуты, а конденсаторы $C_1$ и $C_2$ заряжены до
  разности потенциалов $U_1$ и $U_2$ соответственно, причём знаки
  зарядов, размещающихся на пластинах конденсаторов, показаны на
  рисунке. Какой заряд пройдёт через сопротивление $R$ при
  одновременном замыкании ключей? }
\begin{center}
  \begin{tikzpicture}[circuit ee IEC, thick]
    \node[contact] (1) at (0,0) {};
    \node[contact] (2) at (0,-2) {};
    \draw (1) to[resistor={info={$R_2$}}] ++(right:1.5cm)
    to[capacitor={info={$C_2$}, info=north west:{$-$},info=north
      east:{$+$}}] ++(down:1cm)  
    to[make contact] ++(down:1cm) to[resistor={info={$R_1$}}] (2) to[resistor={info={$R_2$}}]
    ++(left:1.5cm) to[make contact] ++(up:1cm)
    to[capacitor={info={$C_1$}, info=north west:{$-$},info=north
      east:{$+$}}] 
    ++(up:1cm) to[resistor={info={$R_1$}}] (1);
    \draw (1) to[resistor={info={$R$}}] (2); 
  \end{tikzpicture}
\end{center}

\taskpic[5cm]{ В электрической схеме, изображённой на рисунке, в начальный
  момент ключ \textbf{K} разомкнут, а конденсатор не
  заряжен. Параметры схемы указаны на рисунке. Определите начальные
  токи через ключ и через батарею сразу после замыкания ключа.  }
{
  \begin{tikzpicture}[circuit ee IEC, thick]
    \node[contact] (1) at (0,0) {};
    \node[contact] (2) at (0,-1.5) {};
    \node[contact] (3) at (0,-3) {};
    \node[contact] (4) at (1.5,-1.5) {};
    \draw (1) to [resistor={info={$R$}}] (2)
    to[capacitor={info={$C$}}] (3); 
    \draw (2) to[make contact] (4);
    \draw (1) -- ++(right:1.5cm) to[resistor={info={$2R$}}] (4)
    to[resistor={info={$3R$}}] ++(down:1.5cm) -- (3);
    \draw (1) -- (left:1cm) to[battery={info'={$\mathcal{E}, r$}}] ++(down:3cm) -- (3); 
  \end{tikzpicture}
}

\task{ В схеме на рисунке ключи \textbf{K1} и \textbf{K2} разомкнуты,
  а конденсаторы не заряжены. Ключ \textbf{K1} замыкают, оставляя
  \textbf{K2} разомкнутым. 1) Какие напряжения установятся на
  конденсаторах? 2) Какой заряд протечёт через ключ \textbf{K2}, если
  его замкнуть (при замкнутом ключе \textbf{K1})? Параметры схемы
  указаны на рисунке. }
\begin{center}
  \begin{tikzpicture}[circuit ee IEC, thick]
    \node[contact] (1) at (0,0) {};
    \node[contact] (2) at (1.5,0) {};
    \node[contact] (3) at (3,0) {};
    \node[contact] (4) at (1.5,1.5) {};
    \draw (1) to[resistor={info'={$R$}}] (2) to[resistor={info'={$2R$}}] (3);
    \draw (2) to[make contact] (4);
    \draw (1) to[capacitor={info={$C$}}] (4);
    \draw (3) to[capacitor={info'={$2C$}}] (4);
    \draw (1) -- ++(down:1.5cm) to[make contact={info'={\textbf{K1}}}] ++(right:1.5cm)
    to[battery={info'={$\mathcal{E}, r$}}] ++(right:1.5cm) -- (3); 
  \end{tikzpicture}
\end{center}



\end{document}


%%% Local Variables: 
%%% mode: latex
%%% TeX-engine:xetex
%%% TeX-PDF-mode: t
%%% End:
