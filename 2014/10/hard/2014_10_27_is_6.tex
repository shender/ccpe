\input{../../input/main}

\begin{document}

\begin{center}
  \Large{\textbf{Городской центр физического образования, 10 класс.}\\
  \textit{Серия 6Ш, 27 октября 2014.}}
\end{center}

\begin{center}
  \Large\textbf{ Газовые законы + механика = ? }
\end{center}

\Large

\task{ Воздушный шарик, вынесенный из тёплой комнаты ($t_1 =
  27^{\circ}\,C$) на мороз ($t_2 = -23^{\circ}\,C$) некоторое время
  свободно плавает в воздухе. Определите массу резиновой оболочки
  шарика. Его диаметр $d = 40 $ см, молярная масса воздуха $M = 29$
  г/моль, атмосферное давление $p=10^5$ Па. Упругостью оболочки можно
  пренебречь.  }
% Квант, 1990-08

\task{ В сообщающиеся цилиндрические сосуды одинаковых размеров, один
  из которых запаян, а второй открыт, налита ртуть. Уровни ртути в
  сосудах одинаковы, длина части запаянного сосуда, заполненной
  воздухом, равна $l_0$, атмосферное давление, измеренное в
  миллиметрах ртутного столба, равно $H$. Какой станет разность
  уровней ртути в сосудах, если абсолютную температуру воздуха в
  запаянном сосуде увеличить в два раза? }
% Квант, 1990-08

\task{ В вертикальном сосуде объёмом $V$ под тяжёлым поршнем находится
  газ при температуре $T$. Масса поршня $M$, его площадь $S$. Для
  повышения температуры газа на $\Delta T$ градусов ему было сообщено
  количество теплоты $Q$. Найдите изменение внутренней энергии
  газа. Атмосферное давление равно $p_0$, ускорение свободного падения
  $g$. Трение не учитывать. }
% Квант, 1986-04

\task{ Над идеальным двухатомным газом совершают процесс, в котором
  давление и объём газа связаны соотношением $p = \alpha V$. Чему
  равна молярная теплоёмкость газа при его расширении в таком
  процессе? Молярная теплоёмкость при постоянном объёме $C_V = 5/2 \cdot
  R$. }
% Квант, 1986-04


\end{document}


%%% Local Variables: 
%%% mode: latex
%%% TeX-engine:xetex
%%% TeX-PDF-mode: t
%%% End:
