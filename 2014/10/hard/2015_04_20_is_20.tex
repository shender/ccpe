\documentclass[a4paper, 12pt]{article}
% math symbols
\usepackage{amssymb}
\usepackage{amsmath}
\usepackage{mathrsfs}
\usepackage{physsummer}


\usepackage{enumitem}
\usepackage[margin = 2cm]{geometry}

\tolerance = 1000
\emergencystretch = 0.74cm



\pagestyle{empty}
\parindent = 0mm

\begin{document}

\begin{center}
  \Large{\textbf{Городской центр физического образования, 10 класс.}\\
  \textit{Серия 20Ш, 20 апреля 2015.}}
\end{center}

\begin{center}
  \Large\textbf{ Пост--екатеринбургское. }
\end{center}

\Large

\taskpic{ Проводники 1 и 2 лежат в плоскости, перпендикулярной
  однородному магнитному полю. По проводникам текут одинаковые
  токи. Докажите, что на проводники действуют одинаковые (по модулю и
  направлению) силы Ампера. }
{
  \begin{tikzpicture}
    \draw[thick,marrow] (0,0) node[below] {$A_1$} -- (2,0.5) node[below]
    {$A_2$} node[midway,below,blue] {1};
    \draw[thick,marrow] (0,0) to[out=89,in=91] (2,0.5);
    \draw (1,1) node[above,blue] {2};
    \draw[very thick] (2,1.5) ++(right:0.3cm) ++(down:0.3cm) --
    (2,1.5);
    \draw[very thick] (2.3,1.5) ++(left:0.3cm) ++(down:0.3cm) --
    (2.3,1.5);
    \draw (2.3,1.4) node[right] {$\vec{B}$}; 
  \end{tikzpicture}
}

\task{ Электрон влетает в однородное магнитное поле со скоростью $v$
  под углом $\alpha$ к направлению поля. Магнитная индукция поля
  $\vec{B}$. По какой траектории будет двигаться электрон?
  \textit{Указание.} Разложите скорость на составляющие. }

\task{ Медный стержень длины $L$ движется поступательно со скоростью
  $V_0$ в перпендикулярном к себе направлении. Всё это происходит в
  магнитном поле с индукцией $B_0$, вектор магнитной индукции
  перпендикулярен как самому стержню, так и вектору его
  скорости. Найти разность потенциалов между концами стержня. }

\end{document}


%%% Local Variables: 
%%% mode: latex
%%% TeX-engine:xetex
%%% TeX-PDF-mode: t
%%% End:
