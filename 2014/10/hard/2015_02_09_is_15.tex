\input{../../input/main}

\begin{document}

\begin{center}
  \Large{\textbf{Городской центр физического образования, 10 класс.}\\
  \textit{Серия 15Ш, 9 февраля 2015.}}
\end{center}

\begin{center}
  \Large\textbf{ Завершая оптику. }
\end{center}

\Large

\task{ Тонкая плосковыпуклая линза с оптической силой $D$ вставлена в
  стенку аквариума плоской стороной к воде.  Найдите фокусные
  расстояния получившейся системы.  Показатель преломления воды равен
  $n$. }

\task{ На достаточно удалённые предметы смотрят через собирающую линзу
  с фокусным расстоянием $F=9$ см, располагая глаз на расстоянии
  $a=36$ см от линзы. Оцените минимальный размер экрана, который нужно
  расположить за линзой так, чтобы он перекрыл всё поле изображения.
  Считайте, что радиус зрачка равен $r=1{,}5$ мм.}

\taskpic{ Параллельный пучок падает на боковую поверхность стеклянной
  призмы, сечение которой является правильным шестиугольником
  (см. рис.). Точки $A$ и $B$ на рисунке являются серединами
  соответствующих сторон. Пучок преломляется так, что из призмы
  выходят два отдельных параллельных пучка. Найдите минимальный
  показатель преломления материала призмы, при котором такое возможно.
} {
  \begin{tikzpicture}
    \draw[white,fill=blue!20] (-2.5,-0.6) rectangle (-0.75,0.6);
    \node at (0,0) [draw,fill=white,regular polygon, regular
    polygon sides=6, minimum size=2.5cm,outer sep=0pt] {};
    \draw[thick,marrow] (-2.5,-0.6) -- (-0.9,-0.6) node[right,blue] {$B$};
    \draw[thick,marrow] (-2.5,0.6) -- (-0.9,0.6) node[right,blue] {$A$};
  \end{tikzpicture}
}
% Physics Challengs, TPT, May 2012


\end{document}


%%% Local Variables: 
%%% mode: latex
%%% TeX-engine:xetex
%%% TeX-PDF-mode: t
%%% End:
