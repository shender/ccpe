\input{../../input/main}

\begin{document}

\begin{center}
  \Large{\textbf{Городской центр физического образования, 10 класс.}\\
  \textit{Серия 3Ш, 6 октября 2014.}}
\end{center}

\Large

\begin{center}
  \Large \textbf{Для ускорения.}
\end{center}

\task{ На гладкой горизонтальной плоскости покоится доска массой
  $m_1$. На доску со скоростью $v$ въезжает шайба массой $m_2$
  (см. рис.). Какой должна быть минимальная длина доски $l$, чтобы
  шайба не соскользнула с неё? Коэффициент трения скольжения между
  шайбой и доской $\mu$, размер шайбы мал по сравнению с длиной
  доски.}

\begin{center}
  \begin{tikzpicture}
    \draw[thick,interface] (3.8,0) -- (2,0) (2,0.4) -- (0.2,0.4);
    \draw[thick] (2,0) -- (2,0.4);
    \draw[thick,fill=gray!20] (2,0) rectangle (3.2,0.4) node[midway]
    {\small $m_1$};
    \draw[thick] (1,0.4) rectangle (1.3,0.7) node[midway,above=0.1cm] {\small
      $m_2$};
    \draw[thick,->] (1.3,0.55) -- ++(0.5,0) node[above] {$v$};
  \end{tikzpicture}
\end{center}

% Квант, 1998-1

\begin{center}
  \Large\textbf{Теорема об изменении кинетической энергии.}
\end{center}

\Large

\task{ В водоёме укреплена вертикальная труба с гладкой внутренней
  поверхностью, вдоль которой герметично может скользить лёгкий
  поршень. Нижний конец трубы погружен в воду. Поршень, лежавший
  вначале на поверхности воды, медленно поднимают на высоту
  $H$. Найдите работу, которую необходимо при этом совершить. Площадь
  поршня $S$, атмосферное давление $p_0$, плотность воды
  $\rho$. Давлением насыщенных паров пренебречь. }

% Квант, 1998-1


% \task{ Чтобы затащить на горку санки массой $m$, прикладывая
%   постоянную силу вдоль наклонной плоской поверхности горки,
%   необходимо совершить работу не менее $A$. С какой скоростью
%   достигнет основания горки девочка на этих санках, если она съедет с
%   горки с нулевой начальной скоростью по кратчайшему пути? Угол
%   наклона горки к горизонту $\alpha$. Коэффициент трения
%   скольжения между санками и горкой $\mu$.  }

% Квант, 1998-1

\task{ Лыжник съезжает с нулевой начальной скоростью со склона холма
  по прямой, составляющей некоторый угол с горизонтом, и, проехав по
  склону расстояние $s_0$, останавливается, увязнув в снегу. Сила
  сопротивления со стороны снега пропорциональна пройденному пути с
  коэффициентом $k$. Найдите величину максимальной скорости лыжника
  при спуске, если его масса $m$. }

% % Квант, 1998-1

\end{document}


%%% Local Variables: 
%%% mode: latex
%%% TeX-engine:xetex
%%% TeX-PDF-mode: t
%%% End:
