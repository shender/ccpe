\documentclass[a4paper, 12pt]{article}
% math symbols
\usepackage{amssymb}
\usepackage{amsmath}
\usepackage{mathrsfs}
\usepackage{physsummer}


\usepackage{enumitem}
\usepackage[margin = 2cm]{geometry}

\tolerance = 1000
\emergencystretch = 0.74cm



\pagestyle{empty}
\parindent = 0mm

\begin{document}

\begin{center}
  \Large{\textbf{Городской центр физического образования, 10 класс.}\\
  \textit{Серия 18Ш, 16 марта 2015.}}
\end{center}

\begin{center}
  \Large\textbf{ Сложные конденсаторы. }
\end{center}

\Large

\task{ Плоский конденсатор ёмкостью $C$ заряжен до разности
  потенциалов $U$. Как изменится разность потенциалов, если заряд
  одной обкладок увеличить в два раза, а заряд второй обкладки не
  менять? }
% Манида, №267

\task{ Имеется схема, состоящая из большого числа произвольно
  соединённых конденсаторов. Увеличится или уменьшится ёмкость схемы,
  если один из проводов разрезать? }
% Манида, №268

\task{ Система из двух одинаковых последовательно соедиённых
  конденсаторов подключена к источнику постоянного напряжения и ЭДС
  $\mathcal{E}$. Ёмкость каждого конденсатора равна $C$. Обкладки
  одного из конденсаторов закорачивают. Найдите изменение энергии
  системы конденсаторов и работу источника. Равны ли эти величины?
  Если нет, то почему? }
% Манида, №272

\task{ Имеется три тонких коаксиальных металлических цилиндра. Малый
  цилиндр несёт на себе заряд $-q$, два других несут на себе
  одинаковые заряды $q/2$. Большой и средний цилиндры соединяют тонкой
  проволокой. Найдите заряды цилиндров после того, как процесс
  перераспределения зарядов кончится. }
% Манида, №266

\end{document}


%%% Local Variables: 
%%% mode: latex
%%% TeX-engine:xetex
%%% TeX-PDF-mode: t
%%% End:
