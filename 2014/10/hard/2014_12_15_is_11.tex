\input{../../input/main}

\begin{document}

\begin{center}
  \Large{\textbf{Городской центр физического образования, 10 класс.}\\
  \textit{Серия 11Ш, 15 декабря 2014.}}
\end{center}

\begin{center}
  \Large\textbf{ Сложная смесь. }
\end{center}

\Large


\task{ При одновременном подключении двух вольтметров последовательно
  к источнику тока показание первого $U_1=6$~В, второго $U_2=8$~В. При
  параллельном подключении тех же вольтметров показания
  $U_3=12$~В. Определите ЭДС источника. }
% Манида, №283

\task{ Теплоизолированный сосуд разделен на две части лёгким
  поршнем. В левой части сосуда находится $m_1=3$~г водорода при
  температуре $T_1=300$~К, в правой части --- $m_2=16$~г кислорода при
  температуре $T_2=400$~К. Поршень слабо проводит тепло, и температура
  в сосуде постепенно выравнивается. Какое количество теплоты отдаст
  кислород к тому моменту, когда поршень перестанет двигаться? }
% Квант, 1985-09, Ф934

\task{ Вода течёт по длинному каналу с прямоугольным сечением,
  наклонённому к горизонту. Можно считать, что сила трения воды о дно
  и берега канала пропорциональна средней скорости потока и обратно
  пропорциональна его глубине. Во время паводка количество воды,
  протекающей через сечение канала за одну секунду, увеличивается
  вдвое. Как меняется при этом средняя скорость потока? }
% Квант, 1983-04, Ф795

\end{document}


%%% Local Variables: 
%%% mode: latex
%%% TeX-engine:xetex
%%% TeX-PDF-mode: t
%%% End:
