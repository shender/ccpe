\documentclass[a4paper, 12pt]{article}
% math symbols
\usepackage{amssymb}
\usepackage{amsmath}
\usepackage{mathrsfs}
\usepackage{physsummer}


\usepackage{enumitem}
\usepackage[margin = 2cm]{geometry}

\tolerance = 1000
\emergencystretch = 0.74cm



\pagestyle{empty}
\parindent = 0mm

\begin{document}

\begin{center}
  \Large{\textbf{Городской центр физического образования, 10 класс.}\\
  \textit{Серия 2Ш, 29 сентября 2014.}}
\end{center}
\large

\begin{center}
  \Large \textbf{Для ускорения.}
\end{center}

\task{ Блок массы $M$ может свободно вращаться относительно своей оси.
  Через блок переброшена очень лёгкая нерастяжимая нить, один конец
  которой закреплён, а к другому привязан груз массы $4M$. С какой
  силой нужно действовать на блок, чтобы его ускорение было направлено
  вверх и составляло по величине $a$? Вся масса блока сосредоточена в
  его оси\footnote{Зачем нужно это условие?}. Свисающие концы нити
  вертикальны. }

\begin{center}
  \Large \textbf{Блоки, нити и клинья.}
\end{center}

\Large

\task{ Грузы $M$ и $m$ при помощи нерастяжимой лёгкой нити подвешены
  на блоке. С каким ускорением нужно двигать блок в вертикальном
  направлении, чтобы ускорения грузов относительно поверхности Земли
  были направлены в одну сторону? }

% Сорос, 97-10

\task{ На гладком клине с углом $\alpha$ при основании находится
  небольшое тело. С каким ускорением нужно двигать клин по
  вертикали, чтобы тело оставалось на одной высоте? Основание клина
  остаётся при движении горизонтальным. }

% Сорос, 98-11

\taskpic{ Найдите ускорение груза \textbf{1} в системе, изображённой
  на рисунке. Горизонтальная плоскость гладкая, трения между грузами
  нет, нить и блоки невесомы, нить нерастяжима, массы всех трёх грузов
  одинаковы. В начальный момент все тела покоятся. Ускорение
  свободного падения равно $g$. } {
  \begin{tikzpicture}
    \draw[interface, very thick] (3.8,0) -- (0.2,0);
    \draw[thick] (0.5,0) rectangle (3,1.5) node[midway] {\normalsize 3};;
    \draw[very thick] (0.5,1.5) -- ++(0,0.5);
    \draw[very thick] (3,1.5) -- ++(0,0.5);
    \draw[very thick] (0.5,2) circle (0.2cm);
    \draw[very thick] (3,2) circle (0.2cm);
    \draw[thick] (1.5,1.5) rectangle (2.5,2) node[midway] {\normalsize
      2};
    \draw[thick] (1.5,1.8) -- (0.5,1.8);
    \draw[thick] (0.5,2.2) -- (3,2.2);
    \draw[thick] (3,1.2) rectangle (3.4,0.6) node[midway] {\normalsize
      1};
    \draw[thick] (3.2,2) -- (3.2,1.2);
  \end{tikzpicture}
}

% Московские олимпиады, 1.58

\end{document}


%%% Local Variables: 
%%% mode: latex
%%% TeX-engine:xetex
%%% TeX-PDF-mode: t
%%% End:
