\input{../../input/main}

\begin{document}

\begin{center}
  \Large{\textbf{Городской центр физического образования.}\\
  \textit{Серия 3Ш, февраль 2015.}}
\end{center}

\begin{center}
  \Large\textbf{ Законы сохранения. }
\end{center}

\large

\task{ На горизонтальной поверхности стоит сосуд с водой, закрытый
  лёгким подвижным поршнем, на котором лежит тяжёлый груз массой
  $M$. У дна сосуда имеется отверстие сечением $s$, через которое
  вытекает вода. Какова установившаяся скорость движения сосуда, если
  сила трения между сосудом и поверхностью пропорциональна скорости
  сосуда, причём коэффициент пропорциональности равен $k$. Площадь
  сечения сосуда $S$. }
% Квант, 1979, Асламазов

\task{ Два шарика, сделанные из одного и того же материала, движутся
  навстречу друг другу со скоростями $v_1$ и $v_2$. На сколько
  возрастёт температура шариков после лобового абсолютно неупругого
  удара, если удельная теплоёмкость шариков $c$? Начальные температуры
  шариков были одинаковыми. }
% Квант, 1989-04, Черноуцан

\task{ Два вагона, массы которых $M_1$ и $M_2$, движутся навстречу
  друг другу со скоростями $v_1$ и $v_2$. При столкновении происходит
  сжатие четырёх одинаковых буферных пружин, после чего вагоны
  расходятся. Найдите максимальную деформацию каждой пружины, если её
  жесткость равна $k$. }
% Квант, 1989-04, Черноуцан

\taskpic{ На бруске длиной $l$ и массой $M$, расположенном на гладкой
  горизонтальной поверхности, лежит маленькое тело массой
  $m$. Коэффициент трения между телом и бруском равен $\mu$. С какой
  скоростью $v$ должна двигаться система, чтобы после упругого удара
  бруска о стенку тело упало с бруска? }
{
  \begin{tikzpicture}
    \draw[thick,interface] (3,2) -- (3,0) -- (0,0);
    \draw[thick] (0.2,0) rectangle ++(1.5,0.5) node[above] {$M$};
    \draw[thick] (0.2,0.5) rectangle ++(0.5,0.5) node[above] {$m$};
    \draw[thick,->] (1.8,0.25) -- ++(0.7,0); 
  \end{tikzpicture}
}
% Квант, 1989-04, Черноуцан

\taskpic{ В брусок массой $M$, висящий на параллельных нитях длиной
  $l$, попадает горизонтально летящая пуля массой $m$ и застревает в
  нём. В результате удара каждая нить отклоняется на угол
  $\alpha$. Найдите начальную скорость пули $v$. Нити считать
  идеальными (невесомыми и нерастяжимыми). }
{
  \begin{tikzpicture}
    \draw[thick,interface] (0,0) -- (3,0);
    \draw[thick] (1,0) -- ++(0,-2) node[midway,left] {$l$};
    \draw[thick] (2.5,0) -- ++(0,-2);
    \draw[thick] (0.8,-2) rectangle ++(1.9,-0.4)
    node[midway,above=0.2cm] {$M$}; 
    \draw[fill=black] (0,-2.2) circle (0.08cm) node[above] {$m$};
    \draw[thick,->] (0.1,-2.2) -- ++(0.5,0); 
  \end{tikzpicture}
}
% Квант, 1989-04, Черноуцан


\end{document}


%%% Local Variables: 
%%% mode: latex
%%% TeX-engine:xetex
%%% TeX-PDF-mode: t
%%% End:
