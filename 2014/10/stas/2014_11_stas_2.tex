\input{../../input/main}

\begin{document}

\begin{center}
  \Large{\textbf{Городской центр физического образования.}\\
  \textit{Серия 2С, ноябрь 2014.}}
\end{center}

\Large

\task{ Имеется конденсатор ёмкостью $C$ и десять аккумуляторов с ЭДС
  $\eps=1$ В каждый. Как следует заряжать конденсатор до напряжения
  $U=10$ В, чтобы работа источников была минимальной? Каков КПД в
  оптимальном режиме зарядки? Индуктивностью, ёмкостью и активным
  сопротивлением соединительных проводов можно пренебречь. }
% Манида, 282

\task{ Два одинаковых проводящих шарика радиуса $R$ соединены тонкой
  натянутой проволочкой длины $L$ ($L \gg R$). Систему вносят в
  однородное электрическое поле $E_0$, направленное вдоль
  проволочки. Какой заряд протечёт по ней? Какое количество тепла
  выделится в сопротивлении проволочки?   }
% Зильберман, стр. 100

\task{ Плоский конденсатор состоит из двух больших пластин площади $S$
  каждая, расположенных на малом расстоянии $d$ друг от
  друга. Пластины заряжены, их заряды $Q$ и $2Q$. Пластины замыкают
  резистором $R$. Какой заряд протечёт по этому резистору? Сколько в
  нём выделится тепла? }
% Зильберман, ШФО, стр. 101

\task{ Металлическая сфера радиусом $R$ заземлена проводником с
  сопротивлением $r$. Из бесконечности, с нулевой начальной скоростью,
  на сферу движется точечный заряд $q$, имеющий массу $m$. Найти
  количество тепла, выделяющееся в проводе к тому моменту, когда заряд
  попадёт в сферу. }
% Зильберман-Сурков, №158

\end{document}


%%% Local Variables: 
%%% mode: latex
%%% TeX-engine:xetex
%%% TeX-PDF-mode: t
%%% End:
