\input{../../input/main}

\begin{document}

\begin{center}
  \Large{\textbf{Городской центр физического образования.}\\
  \textit{Серия 1С, ноябрь 2014.}}
\end{center}

\Large

\task{ Над одним молем идеального газа совершают процесс, во время
  которого его давление и объём связаны соотношением $(V+V_0)(p+p_0) =
  \mathrm{const}$. Определите максимальную температуру газа во время
  этого процесса. }
% Манида, №220

\task{ Теплоизолированный сосуд откачан до глубокого
  вакуума. Окружающий сосуд одноатомный газ имеет температуру $T_0$. В
  некоторый момент открывают кран и происходит заполнение сосуда
  газом. Какую температуру $T_1$ будет иметь газ в сосуде после его
  заполнения? Считайте, что заполнение сосуда происходит достаточно
  быстро. }
% Слободецкий-Орлов, №304

\task{ В цилиндре объёмом 10 л, закрытом поршнем и помещённом в
  термостате с температурой $40^{\circ}$C, находится по $0.05$ моль
  двух веществ. Определить массу жидкости в цилиндре после
  изотермического сжатия, вследствие которого объём под поршнем
  уменьшается в три раза. При температуре $40^{\circ}$C давление
  насыщенных паров первой жидкости $p_{\mbox{н1}} = 7$ кПа, давление
  второй жидкости $p_{\mbox{н2}} = 17$ кПа. Молярная масса первой
  жидкости составляет $M_1 = 1.8 \cdot 10^{-2}$ кг/моль, второй
  жидкости $M_2 = 4.6 \cdot 16^{-2}$ кг/моль. }
% Слободецкий-Орлов, №306

\task{ В теплоизолированном сосуде заключен одноатомный идеальный газ,
  характеризуемый параметрами $p_0,V_0,T_0$. Сверху газ закрыт тяжёлым
  поршнем массой $M$. В некоторый момент поршень опускают, и он
  начинает падать. Найдите значения $p_1,V_1,T_1$ в тот момент, когда
  ускорение поршня равно нулю. Площадь поршня равна $S$, атмосферным
  давлением можно пренебречь. }
% Манида, №195

\end{document}


%%% Local Variables: 
%%% mode: latex
%%% TeX-engine:xetex
%%% TeX-PDF-mode: t
%%% End:
