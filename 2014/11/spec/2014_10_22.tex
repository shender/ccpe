\documentclass[a4paper, 12pt]{article}
% math symbols
\usepackage{amssymb}
\usepackage{amsmath}
\usepackage{mathrsfs}
\usepackage{physsummer}


\usepackage{enumitem}
\usepackage[margin = 2cm]{geometry}

\tolerance = 1000
\emergencystretch = 0.74cm



\pagestyle{empty}
\parindent = 0mm

\begin{document}

\begin{center}
  \Large{\textbf{11 класс.}\\
  \textit{22 октября 2014.}}
\end{center}


\begin{center}
  \Large \textbf{Гармонические колебания.}
\end{center}

\large

\taskpic{ Тонкий обруч радиусом $r$ катается без скольжения по
  внутренней поверхности цилиндра радиусом $R$, совершая малые
  колебания около положения равновесия. Найдите период этих
  колебаний. }
{
  \begin{tikzpicture}
    \draw[thick,interface] (0,0) arc (-30:-150:1.5cm);
    \draw[thick] (0,0) ++ (150:1.5cm) ++ (-60:1.1cm) circle (0.4cm);
    \draw[thick,blue,->] (0,0) ++ (150:1.5cm) --++ (-130:1.5cm)
    node[midway,fill=white] {$R$};  
  \end{tikzpicture}
}
% Квант, 1981-03

\task{ В вертикально расположенном цилиндрическом сосуде с площадью
  сечения $S$ может перемещаться без трения массивный поршень массой
  $M$. Сосуд заполнен газом. В положении равновесия расстояние между
  поршнем и дном сосуда равно $h$. Определите период малых колебаний,
  которые возникают при отклонении поршня из положения
  равновесия. Атмосферное давление равно $p_0$, газ идеальный,
  температура газа постоянна. }
% Квант, 1981-03

\task{ На очень шероховатый цилиндр радиуса $r$, расположенный
  горизонтально, надет тонкий обруч радиуса $R$. Найти период
  колебаний обруча в вертикальной плоскости. }
% Квант, Ф751, 1982-10

\task{ Три шарика массой $m$ каждый соединены в треугольник при помощи
  трёх одинаковых пружин, каждая жёсткости $k$. Шарикам придают равные
  скорости, направленные к центру треугольника. Определите период
  колебаний системы. }

\taskpic{ Как изменится частота колебаний маятника, представляющего
  собой груз на лёгком стержне, если к середине стержня прикрепить
  горизонтальную пружину жёсткости $k$? На рисунке изображено
  состояние равновесия.}
{
  \begin{tikzpicture}
    \draw[thick,interface] (1,3) -- (2.1,3);
    \draw[thick] (1.5,3) -- (1.5,0) node[left,midway,blue] {$l$};
    \draw[fill=black] (1.5,0) circle (0.1cm) node[right,blue] {$m$};
    \draw[thick,spring] (1.5,1.5) -- (2.5,1.5) node[above,midway,blue]
    {$k$}; 
    \draw[thick,interface] (2.5,1.9) -- (2.5,1.1);
  \end{tikzpicture}
}

\taskpic{ К наклонной стене подвешен маятник длиной $l$. Маятник
  отклонили от вертикали на малый угол, в два раза превышающий угол
  наклона стены к вертикали, и отпустили. Найдите период колебаний
  маятника, если удары о стену абсолютно упругие. }
{
  \begin{tikzpicture}
    \draw[thick,interface] (0,0) -- ++(80:3cm);
    \draw[thick,fill=black] (0,0) ++(80:3cm) -- ++(-70:2.8cm)
    node[midway,blue,above=0.2cm] {$l$} circle (0.1cm);
    \draw[dashed] (0,0) ++(80:3cm) -- ++(-90:3cm);
    \draw[blue,->] (-0.2,1) node[left] {$\alpha$} to[out=0,in=-90]
    (0.42cm,1.8cm); 
    \draw[blue,->] (1.4cm,1cm) node[right] {$2\alpha$}
    to[out=180,in=-90] (0.7cm,1.8cm); 
  \end{tikzpicture}
}

\end{document}


%%% Local Variables: 
%%% mode: latex
%%% TeX-engine:xetex
%%% TeX-PDF-mode: t
%%% End:
