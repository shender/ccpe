\documentclass[a4paper, 12pt]{article}
% math symbols
\usepackage{amssymb}
\usepackage{amsmath}
\usepackage{mathrsfs}
\usepackage{physsummer}


\usepackage{enumitem}
\usepackage[margin = 2cm]{geometry}

\tolerance = 1000
\emergencystretch = 0.74cm



\pagestyle{empty}
\parindent = 0mm

\begin{document}

\begin{center}
  \Large{\textbf{11 класс.}\\
  \textit{10 декабря 2014.}}
\end{center}


\begin{center}
  \Large \textbf{ Переменный ток. }
\end{center}

\large

\task{ В тетраэдре \textbf{ABCD} рёбра составлены из катушек
  индуктивности: $AB=L, BC=6L, AC=3L,AD=2L,CD=5L,BD=4L$. К вершинам
  \textbf{AB} подсоединили последовательно соединённые резистор
  сопротивлением $R=100$ Ом, батарейку с ЭДС $\mathcal{E} = 4.6$ В,
  миллиамперметр и ключ. Индуктивность катушки $L=1$ мГн. Взаимной
  индуктивностью катушек пренебречь.
  \begin{enumerate}
  \item Вычислите силу тока, протекающую через миллиамперметр спустя 1
    минуту после замыкания ключа. 
  \item Вычислите силу тока, протекающего через каждую из катушек в
    тот момент, когда сила тока, протекающего через миллиамперметр,
    равна $I_A=23$ мА. 
  \end{enumerate}
}
% Регион-2014, 11 класс

\taskpic{ В схеме, изображённой на рисунке, в начальный момент ключ
  разомкнут. Определите ток в ветвях цепи сразу после замыкания ключа.
  Найдите установившиеся токи в цепи после окончания переходного
  процесса. Внутренним сопротивлением батареи пренебречь. }
{
  \begin{tikzpicture}[circuit ee IEC,thick]
    \node[contact] (1) at (0,0) {};
    \node[contact] (2) at (0,2) {};
    \draw[thick] (1) --++(0.5,0) to[inductor={info'={$L$}}] ++(0,2) -- (2);
    \draw[thick] (1) --++(-0.5,0) to[resistor={info={$R_1$}}] ++(0,2) -- (2);
    \draw[thick] (2) --++(0,0.2) -- ++(-2,0)
    to[battery={info={$\mathcal{E}$}}] ++(0,-3.5) to[make contact]
    ++(2,0) to[resistor={info'={$R_2$}}] (1);  
  \end{tikzpicture}
}
% Можаев, Переходные процессы в электрических цепях, №7


\task{ В колебательном контуре при разомкнутом ключе происходят
  затухающие колебания тока. В тот момент, когда ток в контуре
  достигает максимального значения и равен $I_{max}$, замыкают
  ключ. Определите количество теплоты, которое выделится в резисторе
  после замыкания ключа. }
\begin{center}
  \begin{tikzpicture}[circuit ee IEC,thick]
    \node[contact] (1) at (0,0) {};
    \node[contact] (2) at (3,0) {};
    \draw[thick] (1) to[inductor={info={$L_1$}}] (2);
    \draw[thick] (1) to[make contact] ++(0,-2)
    to[inductor={info={$L_2$}}] ++(3,0) -- (2);
    \draw[thick] (1) -- ++(-1,0) to[resistor={info={$R$}}] ++(0,2)
    to[capacitor={info'={$C$}}] ++(5,0) -- ++(0,-2) -- (2); 
  \end{tikzpicture}
\end{center}
% Можаев, Переходные процессы в электрических цепях, №9

\end{document}


%%% Local Variables: 
%%% mode: latex
%%% TeX-engine:xetex
%%% TeX-PDF-mode: t
%%% End:
