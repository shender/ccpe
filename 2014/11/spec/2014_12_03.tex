\documentclass[a4paper, 12pt]{article}
% math symbols
\usepackage{amssymb}
\usepackage{amsmath}
\usepackage{mathrsfs}
\usepackage{physsummer}


\usepackage{enumitem}
\usepackage[margin = 2cm]{geometry}

\tolerance = 1000
\emergencystretch = 0.74cm



\pagestyle{empty}
\parindent = 0mm

\begin{document}

\begin{center}
  \Large{\textbf{11 класс.}\\
  \textit{3 декабря 2014.}}
\end{center}


\begin{center}
  \Large \textbf{ Повторение термодинамики. }
\end{center}

\large

\taskpic{ В архиве Кельвина рукопись с $(p,V)$ диаграммой, на которой
  был изображён циклический процесс в виде прямоугольного треугольника
  \textbf{ACB}. Угол $C$ был прямым, а в точке \textbf{K}, лежащей на
  середине стороны \textbf{AB}, теплоёмкость многоатомного газа
  $\mathrm{CH}_4$ обращалась в ноль. Газ можно считать идеальным. От
  времени чернила выцвели, и на рисунке остались видны только
  координатные оси и точки \textbf{С} и \textbf{K}. С помощью циркуля
  и линейки без делений восстановите положение треугольника
  \textbf{ACB}. Известно, что в точке \textbf{A} объём был меньше, чем
  в \textbf{B}. }
{
  \begin{tikzpicture}
    \draw[very thick,->] (0,0) node[below] {0} -- (3.5,0) node[above] {$V$};
    \draw[very thick,->] (0,0) -- (0,4) node[right] {$p$};
    \draw[fill=black] (0,0.3) ++(30:1.2cm) circle (0.05cm) node[above]
    {\textbf{C}};
    \draw[fill=black] (0,0.3) ++(30:2.5cm) circle (0.05cm) node[above]
    {\textbf{K}}; 
  \end{tikzpicture}
}

\taskpic{ В архиве Кельвина нашли рукопись, на которой был изображён
  процесс $1 \to 2 \to 3$, совершённый над одним молем азота. От
  времени чернила выцвели, и стало невозможно разглядеть, где
  находятся оси давления и объёма. Однако из текста следовало, что
  состояния 1 и 3 лежат на одной изохоре, а также то, что в процессах
  $1 \to 2$ и $2 \to 3$ объём газа изменяется на $\Delta V$. Кроме
  того, было сказано, что количество теплоты, подведённой в процессе
  $1 \to 2 \to 3$ к $\mathrm{N}_2$ равно нулю. Определите, на каком
  расстоянии (в единицах объёма) от оси давлений находится изохора,
  проходящая через точки 1 и 3. }
{
  \begin{tikzpicture}
    \draw[very thick, marrow] (0,0) node[below] {\textbf{1}} --
    ++(40:3cm) node[right] {\textbf{2}};
    \draw[very thick, marrow] (0,0) ++ (40:3cm) -- (0,1.2) node[above]
    {\textbf{3}}; 
  \end{tikzpicture}
}


\task{ Камеру объёмом $V=10$ л наполнили сухим воздухом, ввели в неё
  $m=3$ г воды, закрыли, а затем нагрели до
  $t_2=100^{\circ}\,$C. Какое давление установится в камере, если
  первоначальное давление было $p_1=10^5$ Па? }
% Квант, 1992-06, стр. 58

\task{  Компрессор, изначально предназначенный для сжатия воздуха,
  используется для сжатия гелия. Обнаружилось, что компрессор
  перегревается. Объясните этот эффект, предполагая, что процесс
  сжатия~---~адиабатический, а начальные давления в обоих газах
  равны. }
% Wisconsin, 1014



\end{document}


%%% Local Variables: 
%%% mode: latex
%%% TeX-engine:xetex
%%% TeX-PDF-mode: t
%%% End:
