\documentclass[a4paper, 12pt]{article}
% math symbols
\usepackage{amssymb}
\usepackage{amsmath}
\usepackage{mathrsfs}
\usepackage{physsummer}


\usepackage{enumitem}
\usepackage[margin = 2cm]{geometry}

\tolerance = 1000
\emergencystretch = 0.74cm



\pagestyle{empty}
\parindent = 0mm

\begin{document}

\begin{center}
  \Large{\textbf{11 класс.}\\
  \textit{15 октября 2014.}}
\end{center}


\begin{center}
  \Large \textbf{Распределённая масса.}
\end{center}

\large


\task{ Тонкое верёвочное кольцо массой $m$ и радиусом $R$ положили на
  гладкую горизонтальную поверхность и раскрутили до угловой скорости
  $\omega$. Найдите силу натяжения верёвки. }

\task{ Струя воды сечением $S$ ударяется о стенку, расположенную
  перпендикулярно струе. Скорость воды в струе $v$, после удара вода
  теряет скорость и стекает по стенке. Какова сила давления воды на
  стенку? Плотность воды $\rho$.  }

\task{ Тонкая цепочка длиной $l$ и массой $m$ удерживается за верхний
  конец так, что нижним концом она касается земли. Цепочку отпускают,
  и она начинает падать. Найдите силу давления цепочки на землю через
  время $t$. Цепочка неупругая и мягкая. }

\taskpic{ Длинная тонкая цепочка перекинута через блок так, что её правая
  часть свисает до пола, а левая лежит, свернувшись клубком, на уступе
  высотой $H$. Цепочку отпускают, и она приходит в движение. Найдите
  установившуюся скорость движения цепочки. Блок идеальный, цепочка
  неупругая. }
{
  \begin{tikzpicture}
    \draw[very thick, interface] (1,3) -- (3,3);
    \draw[thick] (2,3) -- (2,2);
    \draw[very thick] (2,2) circle (0.5cm);
    \draw[thick] (1.5,2) -- (1.5,0.7);
    \draw[thick] (2.5,2) -- (2.5,0);
    \draw[very thick] (1,0.7) -- (2,0.7) -- (2,0) -- (3,0);
    \draw[thick,blue,<->] (1.2,0.7) -- (1.2,0) node[midway,left]
    {$H$}; 
  \end{tikzpicture}
}

\task{ Цепочку массой $m$ и длиной $l$ подвесили за концы к
  потолку. При этом оказалось, что в местах закрепления цепочка
  образует углы $\alpha$ с вертикалью. Найдите расстояние $h$ от
  нижней точки цепочки до потолка. }

\task{ Верёвку длиной $l$ закрепили за концы на разных
  уровнях. Оказалось, что у одного конца верёвка образует с
  вертикалью угол $\alpha$, а у другого~---~угол $\beta$. На сколько
  первый конец верёвки выше второго? }

\end{document}


%%% Local Variables: 
%%% mode: latex
%%% TeX-engine:xetex
%%% TeX-PDF-mode: t
%%% End:
