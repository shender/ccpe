\input{../../input/main}

\begin{document}

\begin{center}
  \Large{\textbf{11 класс.}\\
  \textit{26 ноября 2014.}}
\end{center}


\begin{center}
  \Large \textbf{ Цепи и переходные процессы. }
\end{center}

\large

\taskpic{ Электрическая цепь состоит из конденсатора ёмкостью $C=125$
  мкФ, резистора $R$, сопротивление которого неизвестно, источника
  постоянного тока с ЭДС $\mathcal{E}=70$ В и внутренним сопротивлением
  $r=R/2$. Вначале конденсатор не заряжен, ток отсутствует. Ключ К
  замыкают и через некоторое время размыкают. Оказалось, что сразу
  после размыкания ключа сила тока, текущего через конденсатор, в 2
  раза больше силы тока, текущего через конденсатор непосредственно
  перед размыканием ключа. Найдите количество теплоты, которое
  выделилось в цепи после размыкания ключа К. }
{
  \begin{tikzpicture}[circuit ee IEC,thick]
    \node[contact] (1) at (2,0) {};
    \node[contact] (2) at (2,2) {};
    \draw[thick] (1) to[resistor] (2);
    \draw[thick] (1) -- ++(-1.5,0) to[capacitor] ++(0,2) -- (2);
    \draw[thick] (2) to[make contact] ++(1.5,0) to[battery] ++(0,-2)
    -- (1); 
  \end{tikzpicture}
}
% Регион-11, 2014

\task{ Параметры электрической цепи указаны на схеме. Вначале ключ К
  разомкнут.
  \begin{enumerate}
  \item Определите напряжение на конденсаторе ёмкостью $C$. 
  \item Определите силу тока, который потечёт через резистор
    сопротивлением $3R$, сразу после замыкания ключа К. 
  \item Какое напряжение установится на конденсаторе ёмкостью $C$
    после того, как переходные процессы завершатся? 
  \end{enumerate}
}
\begin{figure}[h]
  \centering
  \begin{tikzpicture}[circuit ee IEC,thick]
    \node[contact] (1) at (2,0) {};
    \node[contact] (2) at (2,4) {};
    \draw[thick] (2) to[capacitor={info={$3C$}}] ++(0,-2)
    to[resistor={info={$3R$}}] (1);
    \draw[thick] (2) -- ++(2,0) to[capacitor={info={$2C$}}] 
    ++(0,-1.5) to[resistor={info={$2R$}}]  ++(0,-1.5)
    to[battery={info={$2\mathcal{E}$}}] ++(0,-1);
    \draw[thick] (1) to[make contact={info={К}}] ++(2,0);
    \draw[thick] (2) -- ++(-2,0) to[capacitor={info={$C$}}] 
    ++(0,-1.5) to[resistor={info={$R$}}]  ++(0,-1.5)
    to[battery={info={$\mathcal{E}$}}] ++(0,-1) -- (1);
  \end{tikzpicture}
\end{figure}
% Регион-11, 2013

\end{document}


%%% Local Variables: 
%%% mode: latex
%%% TeX-engine:xetex
%%% TeX-PDF-mode: t
%%% End:
