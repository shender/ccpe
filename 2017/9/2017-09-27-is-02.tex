\input{../../main}

\renewcommand{\libproblempath}{../../../materials/problems_db}

\begin{document}

\setphysstyle{ГЦФО 9}{Серия Ш-02}{27.09.2017}
\setcounter{notask}{4}

\center{\large \textbf{Ещё раз о системах отсчёта} }

\vspace{0.25cm}

\libproblem{shapiro-bodik}{1.1}

\vspace{0.25cm}

\center{\large \textbf{Кое-что о жёстких стержнях} }

\flushleft

Пусть имеется абсолютно жёсткий стержень, который как-то двигается. В
частности, двигаются его концы, точки \textbf{A} и
\textbf{B}. Обозначим их скорости $\vec{v}_A$ и $\vec{v}_B$. 

\begin{figure}[h]
  \centering
  \begin{tikzpicture}
    \draw[very thick] (0,0) node[below] {\textbf{A}} -- (3,4)
    node[left] {\textbf{B}};
    \draw[thick,dashed] (3,4) -- ++(0.6,0.8);
    \draw[blue,thick,->] (0,0) -- ++(90:1.5cm) node[above] {$v_A$};
    \draw[red,thick,->] (3,4) -- ++(10:2cm) node[right] {$v_B$};
    \draw[blue, thick, arcnode={$\alpha$}] (0,0) ++(90:0.75) arc
    (90:atan2(3,4):0.5cm);
    \draw[red, thick, arcnode={$\beta$}] (3.3,4.4) arc (atan2(3,4):10:0.75cm);
  \end{tikzpicture}
  \caption{Абсолютно жёсткий стержень. }
  \label{fig:rod}
\end{figure}

Спроецируем скорости на направление стержня: $v_A \cos \alpha$, $v_B
\cos \beta$. Так как стержень абсолютно жёсткий, то его длина не может
меняться, а значит,
\begin{equation}
  \label{eq:stiff-rod}
  v_A \cos \alpha = v_B \cos \beta. 
\end{equation}
(в противном случае точка \textbf{B} удалялась бы от точки \textbf{A}
(или приближалась бы к ней)). Уравнение \eqref{eq:stiff-rod} можно и
нужно использовать при решении различных задач.

\vspace{0.5cm}

\libproblem{other}{kvant-2012-1-p34}
\libproblem{original}{palochka01}

\end{document}

%%% Local Variables: 
%%% mode: latex
%%% TeX-engine:xetex
%%% TeX-PDF-mode: t
%%% End:
