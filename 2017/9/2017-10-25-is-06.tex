\documentclass[a4paper, 12pt]{article}
% math symbols
\usepackage{amssymb}
\usepackage{amsmath}
\usepackage{mathrsfs}
\usepackage{physsummer}


\usepackage{enumitem}
\usepackage[margin = 2cm]{geometry}

\tolerance = 1000
\emergencystretch = 0.74cm



\pagestyle{empty}
\parindent = 0mm

\renewcommand{\libproblempath}{../../../materials/problems_db}

\begin{document}

\setphysstyle{ГЦФО 9}{Серия Ш-06}{25.10.2017}
\setcounter{notask}{16}

\parindent=10pt

\begin{center}
  \large \textbf{Движение без проскальзывания}
\end{center}

\begin{wrapfigure}{r}{4cm}
  \begin{tikzpicture}
    \draw[thick,interface] (4,0) -- (0,0);
    \draw[thick] (2,1) circle (1cm);
    \draw[fill=black] (3,1) circle (0.05cm) node[right] {$D$};
    \draw[fill=black] (1,1) circle (0.05cm) node[left] {$B$};
    \draw[fill=black] (2,2) circle (0.05cm) node[above] {$C$};
    \draw[fill=black] (2,0) circle (0.05cm) node[above] {$A$};
    \draw[blue,thick,->] (2,1) -- ++(0.7,0) node[above,midway] {$v$};
    \draw[blue,thick,>=stealth,->] (1.5,0.5) arc (225:135:0.7cm)
    node[right] {$\omega$}; 
  \end{tikzpicture}
\end{wrapfigure}

Рассмотрим диск радиуса $r$, катящийся по горизонтальной
плоскости. Скорость центра диска постоянна и равна $v$. Диск двигается
без проскальзывания. Каждая точка диска участвует в двух движениях: в
поступательном движении со скоростью $v$ и во вращательном вокруг
центра с угловой скоростью $\omega$. Отсутствие
\textit{проскальзывания} (иначе эту ситуацию называют иногда
\textit{качением}) говорит нам о том, что точка $A$ диска двигается с
нулевой скоростью относительно земли. Поэтому получаем
$v - \omega r =0$, откуда $\omega = v/r$.

Скорость точки $B$ найти чуть сложнее. Линейная скорость за счёт
вращательного движения равна $\omega r = v$. Она направлена
вертикально вверх. При этом скорость поступательного движения
по-прежнему направлена горизонтально (и по модулю тоже равна
$v$). Получается, что полная скорость равна по модулю $v_B = v\sqrt{2}$ и
направлена под углом $\pi/4$ к горизонту. 

\begin{wrapfigure}{l}{4cm}
  \begin{tikzpicture}
    \draw[thick,interface] (4,0) -- (1,0);
    \draw[thick] (2,1) circle (1cm);
    % точка B
    \draw[thick,->] (1,1) -- ++(0,0.8) node[above,blue] {$\omega r$};
    \draw[thick,->] (1,1) -- ++(0.8,0) node[above,blue] {$v$};
    \draw[thick,->] (1,1) -- ++(45:1.12);
    \draw[fill=black] (1,1) circle (0.05cm) node[left] {$B$};
    % точка D
    \draw[thick,->] (3,1) -- ++(0,-0.8) node[right=-0.1cm,blue] {$\omega r$};
    \draw[thick,->] (3,1) -- ++(0.8,0) node[above,blue] {$v$};
    \draw[thick,->] (3,1) -- ++(-45:1.128);
    \draw[fill=black] (3,1) circle (0.05cm) node[left] {$D$};
  \end{tikzpicture}
\end{wrapfigure}

В точке $D$ всё почти то же самое: разница лишь в том, что
вращательная скорость смотрит вертикально вниз. Как следствие,
суммарная скорость направлена под углом $-\pi/4$ к горизонту. В точке
$C$ вращательная и поступательная скорость направлены в одну сторону и
параллельно друг другу, поэтому суммарная скорость будет
горизонтальна и равна по модулю $2v$. 

Из всего этого следует, что движение диска можно рассматривать как
вращение относительно оси, проходящей через точку $A$. Такая ось
называется \textit{мгновенной осью вращения}. 

\vspace{1cm}
\parindent=0pt

\libproblem{other}{kvant-1972-09-p52}

\libproblem{manida}{180}

\libproblem{moscow(1968-1985)}{1.16}
% неправильный ответ!!!

\end{document}

%%% Local Variables: 
%%% mode: latex
%%% TeX-engine:xetex
%%% TeX-PDF-mode: t
%%% End:
