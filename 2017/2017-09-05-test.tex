\input{../main}

\renewcommand{\libproblempath}{../../camp/materials/problems_db}

\begin{document}

\large

\setphysstyle{ГЦФО 8}{Вступительная олимпиада}{05.09.2017}

\task{Вагон шириной 2,4 метра движущийся со скоростью 15 м/с был
  пробит пулей летевшей перпендикулярно направлению движения
  вагона. Смещение отверстий в стенах вагона относительно друг друга
  равно 6 см. Найдите скорость пули.}

\task{В прямой цилиндрический сосуд, площадь основания которого
  $S = 100$ см$^2$, наливают 1 л соленой воды плотности
  $\rho_1 = 1{,}15$ г/см$^3$, и опускают льдинку из пресной
  воды. Масса льдинки $m = 1$ кг. Определите, как изменится уровень
  воды в сосуде, если половина льдинки растает. Считайте, что при
  растворении соли в воде объем жидкости не изменяется.}

\task{По реке со скоростью $v$ плывут мелкие льдины, которые
  равномерно распределяются по поверхности воды, покрывая ее $n$-ю
  часть. В некотором месте реки образовался затор. В заторе льдины
  полностью покрывают поверхность воды, не нагромождаясь друг на
  друга. С какой скоростью растет граница сплошного льда?}

\task{К балке массой $m_1 = 400$ кг и длиной $l = 7$ м подвешен груз
  массой $m_2 = 700$ кг на расстоянии $a = 2$ м от одного из
  концов. Балка своими концами лежит на опорах. Какова сила давления
  на каждую из опор?}

\clearpage

\setcounter{notask}{1}

\setphysstyle{ГЦФО 9}{Вступительная олимпиада}{05.09.2017}

\task{ На гладкой горизонтальной поверхности на расстоянии $2l$ друг
  от друга неподвижно лежат два шарика, массой $m$ каждый, связанные
  невесомой нерастяжимой нитью длиной $2l$. Среднюю точку $A$ нити
  начинают двигать с постоянной скоростью $V$ в горизонтальном
  направлении, перпендикулярном нити. Какой путь пройдёт точка $A$ до
  момента столкновения шаров? }
% Кондратьев-Уздин, 1.19

\libproblem{kondratyev-uzdin}{3.105}

\taskpic[5.5cm]{ Электрическая цепь состоит из трёх резисторов с известными
  сопротивлениями $R_1=20\unit{Ом}$, $R_2=30\unit{Ом}$,
  $R_4=60\unit{Ом}$, одного резистора с неизвестным сопротивлением
  $R_3$ и одного переменного резистора. При измерении сопротивления
  $R_{AB}$ между точками $A$ и $B$ этой электрической цепи выяснилось,
  что оно не зависит от сопротивления переменного резистора. Найдите
  величины сопротивлений неизвестного резистора $R_3$ и всей цепи
  $R_{AB}$. }
{
  \begin{tikzpicture}[circuit ee IEC, thick]
    \node[contact={info=180:{$A$}}] (A) at (0,0) {};
    \node[contact={info=0:{$B$}}] (B) at (4,0) {};
    \node[contact] (C) at (2,-1) {};
    \node[contact] (D) at (2,1) {};
    \draw (A) -- ++(0,1) to[resistor={info={$R_1$}}] (D)
    to[resistor={info={$R_2$}}] ++(2,0) -- (B) -- ++(0,-1)
    to[resistor={info={$R_4$}}] (C) to[resistor={info={$R_3$}}]
    ++(-2,0) -- (A);
    \draw (C) to[resistor=adjustable] (D); 
  \end{tikzpicture}
}

\end{document}

%%% Local Variables: 
%%% mode: latex
%%% TeX-engine:xetex
%%% TeX-PDF-mode: t
%%% End:
