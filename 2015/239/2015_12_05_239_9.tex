\input{../../main.tex}

\setphysstyle{ФМЛ 239, 9 класс}{Занятие 9}{12 декабря 2015}

\begin{document}

\large
\begin{center}
  \Large{\textbf{О динамометре}}
\end{center}

\task{ Школьный динамометр тянут в разные стороны, приложив к его
  корпусу (первый крючок) и к пружинке (второй крючок) одинаковые по
  величине силы 1 Н. Движется ли динамометр? Что он показывает при
  этом? }

\task{ На гладком горизонтальном столе находятся два тела массами
  $M_1$ и $M_2$, связанные лёгкой нерастяжимой нитью. Тела тянут в
  противоположные стороны силами $F_1$ и $F_2$, направленными
  параллельно нити. Найти силу натяжения нити. }

\task{ В условиях предыдущей задачи вместо нити применили лёгкую
  пружинку. Найти силу натяжения пружинки после того, как движение
  установилось. }

\task{ В условиях предыдущей задачи пружинка оказалась не очень лёгкой
  --- её масса составляет $m$. Одинаковы ли силы натяжения в разных
  частях пружинки при установившемся движении? Найти удлинение
  пружинки в этих условиях, если известно, что при равенстве
  приложенных сил $F_1=F_2=F_0$ пружинка удлинилась на $X_0$. }

\task{ Школьный динамометр тянут в разные стороны, приложив к его
  корпусу (первый крючок) и к пружинке (второй крючок) силы 1 Н и 3
  Н. Движется ли динамометр? Что он показывает при этом? }

\begin{center}
  \Large{\textbf{О блоке}}
\end{center}

\task{ Грузы $M$ и $2M$ подвешены на блоке при помощи лёгкой и
  нерастяжимой нити. Ось блока начинают перемещать по вертикали. Куда
  должно быть направлено ускорение оси блока и каким оно должно быть,
  чтобы тяжёлый груз мог некоторое время (пока нить не <<закончится>>)
  оставаться неподвижным? }
% Зильберман, ШФО, стр. 39

\taskpic{ С какой силой $F$ нужно давить на куб, чтобы удерживать его
  неподвижным? Масса куба $M$, масса груза $m$, груз свисает на
  вертикальной нити и касается куба. Стол гладкий. }
{
  \begin{tikzpicture}
    \draw[thick,interface] (4,0) -- (0,0);
    \draw[thick] (1.5,0) rectangle (3,1.5) node[midway, blue] {$M$};
    \draw[thick] (3,1.5) -- ++(45:0.3cm);
    \draw[thick] (3,1.5) ++(45:0.3cm) circle (0.2cm);
    \draw[thick] (3,1.5) ++(45:0.3cm) ++(right:0.2cm) --
    ++(down:0.7cm) ++(left:0.4cm) rectangle ++(0.8cm,-0.4cm)
    node[midway,blue] {\small $m$};
    \draw[thick] (3,1.5) ++(45:0.3cm) ++(up:0.2cm) -- ++(left:2.5cm);
    \draw[thick,interface] (0.7,1.5) -- (0.7,2.4);
    \draw[very thick,->] (0.7,0.7) node[left,blue] {$F$} -- (1.5,0.7); 
  \end{tikzpicture}
}

\end{document}


%%% Local Variables: 
%%% mode: latex
%%% TeX-engine:xetex
%%% TeX-PDF-mode: t
%%% End:
