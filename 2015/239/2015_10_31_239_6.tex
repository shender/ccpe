\documentclass[a4paper, 12pt]{article}
% math symbols
\usepackage{amssymb}
\usepackage{amsmath}
\usepackage{mathrsfs}
\usepackage{physsummer}


\usepackage{enumitem}
\usepackage[margin = 2cm]{geometry}

\tolerance = 1000
\emergencystretch = 0.74cm



\pagestyle{empty}
\parindent = 0mm

\setphysstyle{ФМЛ 239, 9 класс}{Занятие 6}{31 октября 2015}

\begin{document}

\Large

\begin{center}
  \LARGE{ Движение по окружности }
\end{center}

\task{ Трамвай движется по круговому повороту радиусом $R$ на
  $90^{\circ}$ с постоянным тангенциальным ускорением, причем в начале
  поворота на скорости $V_{0}$ нормальное ускорение по модулю в два
  раза превышало тангенциальное ускорение. Найдите соотношение между
  нормальным $a_n$ и тангенциальным $a_{\tau}$ ускорением при
  завершении поворота. }
% Кондратьев, Уздин, 1.62

\begin{center}
  \LARGE{ Движение по параболе }
\end{center}

\task{ Озорник бросает камень в горизонтально летящую авиамодель,
  причём в момент броска направление скорости камня, характеризуемое
  углом $\alpha$ к горизонту, было как раз на модель. Скорость полета
  авиамодели $u$. Начальная скорость камня $v$. На какой высоте $h$
  летела авиамодель, если камень попал в неё? Где это произошло, на
  восходящем или нисходящем участке траектории камня? }
% Кондратьев, Уздин, 1.88

\begin{center}
  \LARGE{ Задача на оценку }
\end{center}

\task{ Оцените, на сколько дальше спортсмен бросит гранату, если будет
  бросать её с разбега. }
% НГУ-1, 5.1

\end{document}


%%% Local Variables: 
%%% mode: latex
%%% TeX-engine:xetex
%%% TeX-PDF-mode: t
%%% End:
