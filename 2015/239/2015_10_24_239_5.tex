\input{../../main.tex}

\setphysstyle{ФМЛ 239, 9 класс}{Занятие 5 \\ движение по параболе}{24 октября 2015}

\begin{document}

\Large

\task{ Пушка выбрасывает снаряд со скоростью 60 м/с под углом к
  горизонту. Может ли снаряд пролететь 200 м по горизонтали, не
  поднимаясь выше 30 м над точкой выстрела? Сопротивление воздуха
  пренебрежимо мало. }
% Зильберман, 1.19

\task{ Стена имеет высоту $H=30$ м и толщину $L=10$ м. С какой
  минимальной скоростью и под каким углом к горизонту надо бросить
  камень с поверхности, чтобы перебросить его через стену? }
% Зильберман, 1.20

\task{ Тело бросили под углом $\varphi$ к горизонту со скоростью
  $v_0$. За какое время вектор скорости тела повернётся на угол
  $\varphi/2$? }
% Зильберман, 1.43

\task{ Из двух точек, расположенных на одной высоте $h$ над землёй на
  расстоянии $l$ друг от друга, одновременно бросают два камня: один
  вертикально вверх со скоростью $v_1$, другой горизонтально со
  скоростью $v_2$. Каково минимальное расстояние между камнями в
  процессе движения? Начальные скорости камней лежат в одной
  вертикальной плоскости. }
% Черноуцан, Квант-1990-02, 2

\begin{center}
  \LARGE{ \textbf{Окружность и парабола.} }
\end{center}

\task{ Тело брошено со скоростью $v_0$ под углом $\alpha$ к
  горизонту. Траекторией тела, естественно, будет являться
  парабола. Найдите радиус её кривизны в верхней точке. \\
  \textit{Указание}. Радиусом кривизны кривой в точке $A$ называется радиус
  окружности, по дуге которой точка движется в данный момент
  времени. }

\end{document}


%%% Local Variables: 
%%% mode: latex
%%% TeX-engine:xetex
%%% TeX-PDF-mode: t
%%% End:
