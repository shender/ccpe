\input{../../main.tex}

\setphysstyle{ФМЛ 239, 9 класс}{Занятие 1}{19 сентября 2015}

\begin{document}

\begin{center}
  \LARGE{ \textbf{Разминочная олимпиада.} }
\end{center}

\Large

\task{ По прямой дороге, ведущей через поле, медленно едет автобус ---
  его скорость 5 м/с. Вы можете двигаться по полю со скоростью 3 м/с,
  расстояние от вас до дороги в данный момент составляет 30 метров, до
  автобуса --- 50 метров. Как нужно бежать, чтобы попасть в автобус?}
% Зильберман, Школьные физические олимпиады, 1.6

\task{ Электромотор, сопротивление обмотки которого равно $R$,
  питается от источника постоянного напряжения $U$, при этом через
  него протекает ток $I$. Вычислите потребляемую мотором мощность;
  мощность, теряемую на нагрев обмотки; КПД мотора. Проанализируйте
  зависимость указанных величин от тока в моторе. }
% Квант, 1989-08-Коржуев

\task{ Кусок охлаждённого льда поместили в калориметр. В таблице
  приведены результаты измерений температуры содержимого калориметра в
  зависимости от времени. Постройте график зависимости температуры
  льда и воды от времени. На основании экспериментальных данных
  определите удельные теплоёмкости $c_{\mbox{л}}$ льда и воды
  $c_{\mbox{в}}$. Удельная теплота плавления льда $\lambda=330$
  кДж/кг. Теплоёмкостью калориметра пренебречь, подводимую тепловую
  мощность считайте постоянной.  
\begin{center}
  \begin{tabular}{|c|c|c|c|c|c|c|c|c|c|c|}
    \hline
    $\tau$, с & 0 & 5 & 10 & 15 & 20 & 320 & 330 & 340 & 350 & 360\\
    \hline
    $t$, $^{\circ}$C & -4,8 & -2,5 & 0,0 & 0,0 & 0,0 & 0,0 & 0,0 & 0,0 & 2,5 & 4,9\\
    \hline
  \end{tabular}
\end{center}}
% 2009-09, муниципальный этап


\end{document}


%%% Local Variables: 
%%% mode: latex
%%% TeX-engine:xetex
%%% TeX-PDF-mode: t
%%% End:
