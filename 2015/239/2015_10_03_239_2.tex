\input{../../main.tex}

\setphysstyle{ФМЛ 239, 9 класс}{Занятие 2}{3 октября 2015}

\begin{document}

\Large

\task{ На прямолинейном участке пути $AB$ тело двигалось с постоянным
  ускорением. В начале пути скорость равнялась $v_A$, в конце
  $v_B$. Найдите скорость $v_S$ в середине пути. Сравните её со
  скоростью $v_t$, которую тело имело спустя ровно половину времени
  своего движения по участку $AB$. Какая из этих скоростей больше,
  $v_S$ или $v_t$? Ответ обоснуйте.  }
% 2009-09, муниципальный этап


\task{ Гоночный болид движется по прямолинейному участку трассы
  равноускорено.  Скорость болида в конце участка $v_{2} = 98$ м/с,
  скорость в начале участка $v_1 = 40$ м/с.  Какой была скорость
  болида $v_{x}$ на 1/4 пути от начала разгона? }
% 2013-09, муниципальный этап

\task{ Автомобиль едет по шоссе, параллельному железной дороге, с
  постоянным ускорением $a_1=1 \mbox{ м/с}^2$. В некоторый момент
  времени скорость автомобиля равна 72 км/ч. В этот момент в поезде,
  идущем на расстоянии 600 м впереди автомобиля в ту же сторону со
  скоростью 54 км/ч, машинист включает тормозную систему. Далее поезд
  движется с ускорением $a_1=1 \mbox{ м/с}^2$ до остановки. Найдите
  расстояние между поездом и автомобилем в момент остановки поезда.  }
% Манида, 166

\task{ Спортсмен пробегает стометровку за 10 с. Первые 10 м дистанции
  он бежит с постоянным ускорением $a$, а остальную часть
  дистанции~---~с постоянной скоростью $v$. Найдите ускорение $a$ и
  скорость $v$. }
% Манида, 167

\end{document}


%%% Local Variables: 
%%% mode: latex
%%% TeX-engine:xetex
%%% TeX-PDF-mode: t
%%% End:
